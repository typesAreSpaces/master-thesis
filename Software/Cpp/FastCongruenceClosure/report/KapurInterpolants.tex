\section{Interpolation Algorithms for quantifier-free EUF and octagonal formulas}

In this section we discuss the novel algorithms by Prof. Kapur. First of all,
we will explain basic notions of interpolants will the goal to understand
the key ideas for Kapur's algorithms.

\definition{} Given formulas $\alpha, \beta$ for some theory $T$ such that
$\vdash \alpha \rightarrow \beta$, then a formula $\gamma$ is called
an \textit{interpolant} of the pair $(\alpha, \beta)$ if
$\vdash \alpha \rightarrow \gamma$, $\vdash \gamma \rightarrow \beta$,
and $\gamma$ contains only common symbols of $\alpha$ and $\beta$. A theory
$T$ is \textit{interpolating} if for all pair of formulas $(\alpha, \beta)$
such that $\vdash \alpha \rightarrow \beta$, and interpolant $\gamma$ of
$(\alpha, \beta)$ can be computed.

In 1957, Craig \cite{craig1957} proved that any theory in first-order logic
is interpolating. An equivalent formulation of the interpolating
property as suggested by Prof. Kapur in one of his talks is that for a
pair $(\alpha, \beta)$ such that $\alpha \land \beta$ is unsatisfiable
(i.e. $\vdash \alpha \rightarrow \neg \beta$),
a formula $\gamma$ is an interpolant of $(\alpha, \beta)$ if
$\vdash \alpha \rightarrow \gamma$ and $\vdash \neg (\gamma \land \beta)$.

From the latter interpolant formula point of view, we can characterize
Prof. Kapur's approach in both of his algorithm as follows. In hindsight, the basic
idea is to perform a collection of `reductions' to the formula $\alpha \land \beta$
until a fixed point is found. Any sequence of these reductions keep the
invariant that for the `current interpolant formula' $\gamma$ with respect
to the pair $(\alpha, \beta)$ it satisfies $\vdash \alpha \rightarrow \gamma$ and
$\vdash \neg(\gamma \land \beta)$, on the other hand the requirement for
$\gamma$ to only contain common symbols of $\alpha, \beta$ is achieved only
when there are no more sequence of reductions to be performed.

Initially, Prof. Kapur's algorithm choose as initial formula the conjunction
$\gamma = \alpha \land \neg\beta$ as the current interpolant formula. This formula satisfies
the invariant proposed above since $\vdash \alpha \rightarrow \alpha$ (tautology),
and $\vdash \alpha \rightarrow \neg\beta$ (by assumption), thus $\vdash \alpha \rightarrow \gamma$.
Similary, we can see that $(\alpha \land \neg \beta \land \beta)$ is unsatisfiable for any $\alpha, \beta$.
Any sequence of reductions satisfy the invariant below because
$\vdash \gamma \rightarrow \gamma^{'}$, where $\gamma^{'}$ is a reduced formula from $\gamma$.
It is worth mentioning that Kapur's algorithm guarantees to remove all uncommon symbols
in the current interpolanting formula. With the latter fact and the invariant
mentioned below we can conclude that the current interpolating formula will
become an interpolant formula of the pairs $(\alpha, \beta)$.

\subsection{Interpolation algorithm for quantifier-free EUF}

This algorithm requires six steps. The first one (preprocessing) performs a flattening process,
which is similar to a well-known algorithm known as the Tseytin transform \cite{Tseitin1983}.
The equations obtained after performing this process are of the form: $f(c_1, \dots, c_k) = d, c = d, c \neq d$,
where $f$ is a functional symbol and each $c_i$, $c$ and $d$ are constants. It is important to mention
that flattening might introduce new symbols, hence in order to distinguish these symbols if they are common
or uncommon symbol Kapur introduced a definition where a new symbol is common if and only all the symbols
in the other term of the equation are common symbols.

The second step generates equivalence classes using the equations obtained in the last step. Using the
UNION-FIND data structure we keep track of the representatives of these classes, prefering common expressions (
expression with only common symbols) as representatives. The last step might require performing
rotations in the three representation, so the UNION-FIND data structure maintains its properties and running
times. For the congruence relation components the algorithm mentioned above is used. The algorithm then replaces
all terms in the set of expressions by their representative in the equivalence class they belong to. If
we obtained equivalence classes with non-common representatives (by construction
the rest of the elements in this class are non-common as well) we can delete it.

The third step involves eliminating uncommon functions symbols. This happens when a functional symbol
$f$ is non-common. Thus, we take all expressions containing the functional symbol $f$ of the form
$f(c_1, \dots, c_k) = c, f(d_1, \dots, d_k) = d$ and we remove them from the set of expression, introducing
the expression $(c_1 = d_1, \dots, c_k = d_k) \rightarrow c = d$. Since this expression is a Horn clause,
many of the algorithms studied before play a crucial role in this algorithm. A forth step exposes uncommon
constants inside functional symbols. Essentially the same process as in step 3 is done.

The fifth step we eliminate uncommon symbols conditionally, i.e. we include the conditions (body of the
Horn clause) needed for a specific equality (head of the respective Horn clause)
to occurs. There are several cases:

\begin{itemize}
\item The body of Horn clause $C$ consists of common expressions and the head of $C$
  contains an uncommon term $u$ on only one side of the equation. We conditionally eliminate
  $u$ in the rest of Horn clauses that contain $u$. We then eliminate $C$ from the current
  interpolant formula since there is no way to eliminate the uncommon expression $u$.
\item The body of Horn clause $C$ consists of common expressions and the head of $C$
  contains only uncommon terms. Then, we can do the same as in the previous case with the
  difference that we replace the equation in the head of the Horn clause $C$ by a symbol $\top$
  which means satisfiable (true). 
\end{itemize}

The last step only states that the obtained current interpolating formula is the interpolant
of the pair $(\alpha, \beta)$.

\subsection{Interpolation algorithm for octagonal formulas}

Contrary to the last interpolantion algorithm, this algorithm uses the
simplification rules:

\begin{equation*}
  \AxiomC{$a x_i + b y_j \leq c$} \AxiomC{$-a x_i + b^{'} y_k \leq d$}
  \AxiomC{$x_i \text{ is an uncommon symbol}$}
  \TrinaryInfC{$b y_j + b^{'} y_k \leq c + d$}
  \DisplayProof
\end{equation*}

\begin{equation*}
  \AxiomC{$a x + a y \leq a c$}
  \UnaryInfC{$x + y \leq c$}
  \DisplayProof
\end{equation*}

Applying these rules as necessary to eliminate uncommon symbols might take at most $\bigO{n^2}$
steps. Additionally, this algorithm terminates since the algorithm remove one uncommon symbol
at a time, and, assuming there is a finite number of uncommon symbols, then the algorithm
terminates. 