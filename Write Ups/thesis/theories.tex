\section{Mathematical Theories} \label{math_theories}

A theory $\mathcal{T}$ is a collection of formulas that are closed under logical implication, i.e. if $\mathcal{T} \models \psi$ then $\psi \in \mathcal{T}$. This concept is quite relevant for our thesis work since we will focus on two theories, the quantifier-free fragment of the theory of equality with uninterpreted functions (EUF), and the theory of unit two variable per inequality (UTVPI). 

For some theories it is enough to provide a collection of formulas (known as the axioms of the theory). For the case of the theories of interest for the thesis, the axiomatization if the following:

\subsection{Equality with uninterpreted functions}

\begin{definition} \label{euf_axioms}
  Let $\mathfrak{L}_{EUF} = \{ \{\}, \{ = \}, \{ f_1, \dots, f_n \} \}$ be the language of EUF. The axioms of the theory are:
  \begin{itemize}
    \item (Reflexivity) $\forall x . x = x$
    \item (Symmetry) $\forall x . \forall y . x = y \rightarrow y = x$
    \item (Transitivity) $\forall x . \forall y . \forall z. (x = y \land y = z) \rightarrow x = z$
    \item (Congruence) $\forall x_1  \dots \forall x_n . \forall y_1 \dots \forall y_n . (x_1 = y_1 \land \dots \land x_n = y_n) \rightarrow f(x_1, \dots, x_n) = f(y_1, \dots, y_n) $
  \end{itemize}
\end{definition}

We notice that the congruence axiom is not a first-order logic axiom, but rather an axiom-scheme since it is necessary to \emph{instantiate} such axiom for every arity of the function symbols in a given language.

\subsection{Ordered commutative rings}

In order to describe the UTVPI theory we will first introduce the language and theory of an ordered commutative ring.

\begin{definition}
  Let $\mathfrak{L}_{Ord-R} = \{, \{0, 1 \}, \{ = , \leq \}, \{+, -, * \}, \}$ be the 
  language of an ordered commutative ring $R$. The axioms of the theory are:
  \begin{itemize}
    \item $\forall x . \forall y . \forall z . x + (y + z) = (x + y) + z$
    \item $\forall x . \forall y .  x + y = y + x$
    \item $\forall x . x + 0 = x$
    \item $\forall x . x + (- x) = 0$
    \item $\forall x . \forall y . \forall z. x * (y * z) = (x * y) * z$
    \item $\forall x . x * 1 = x$
    \item $\forall x . \forall y .  x * y = y * x$
    \item $\forall x . \forall y . \forall z . x * (y + z) = x * y + x * z$
    \item $\forall x . \forall y . \forall z . (y + z) * x = y*x + z * x$
    \item $\forall x . \forall y . \forall z . x \leq y \rightarrow x + z \leq y + z$
    \item $\forall x . \forall y . (0 \leq x \land 0 \leq y) \rightarrow 0 \leq x * y$.
    \item $0 \neq 1 \land 0 \leq 1$ 
  \end{itemize}
\end{definition}

Section \ref{decision_procedures} discusses computability 
aspects for the theories of interest that are relevant 
for verification.

%%% Local Variables:
%%% mode: latex
%%% TeX-master: "main"
%%% End:
