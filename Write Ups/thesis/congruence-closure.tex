\subsection{Congruence Closure}

TODO.

\subsubsection{Decision Problem}

TODO.

\subsubsection{Decision Procedure}

In \cite{10.1007/978-3-540-32033-3_33}, the authors 
introduced a Union-Find data structure that supports the 
Explanation operation. This operation receives as input 
an equation between constants. If the input equation is 
a consequence of the current equivalence relation defined 
in the Union Find data structure, the Explanation operation 
returns the minimal sequence of equations used to build 
such equivalence relation, otherwise it returns 
`Not provable`. A proper implementation of this algorithm 
extends the traditional Union-Find data structure with 
a \emph{proof-forest}, which consists of an additional 
representation of the underlying equivalence relation that 
does not compress paths whenever a call to the Find 
operation is made. For efficient reasons, the Find 
operation uses the path compression and weighted union.

The main observation in \cite{10.1007/978-3-540-32033-3_33} 
is that, in order to recover an explanation between 
two terms, by traversing the path between the two nodes 
in the proof tree, the last edge in the path guarantees to 
be part of the explanation. Intuitively, this follows because only 
the last Union operation was responsible of merging the 
two classes into one. Hence, we can recursively recover 
the rest of the explanation by recursively traversing 
the subpaths found.

Additionally, the authors in \cite{10.1007/978-3-540-32033-3_33} 
extended the Congruence Closure algorithm 
\cite{10.1007/978-3-540-39813-4_5} using the above data 
structure to provide Explanations for EUF theory. The congruence 
closure algorithm is a simplification of the congruence 
closure algorithm in \cite{10.1145/322217.322228}. The latter 
combines the traditional \emph{pending} and \emph{combine} list 
into one single list, hence removing the initial 
\emph{combination} loop in the algorithm in 
\cite{10.1145/322217.322228}.

TODO. Does this cover DST, NO and explanations?

%%% Local Variables:
%%% mode: latex
%%% TeX-master: "main"
%%% End:
