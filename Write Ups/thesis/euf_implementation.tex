\section{Implementation}

The description of the interpolation algorithm presented in the
previous section suggests an straight forward implementation
of the first two stages. This thesis work considers the following
implementation for the rest of the stages of the algorithm:

\subsection{New optimized conditional elimination step in Kapur's algorithm}

First, we explain a high level ideal on how 
we improve the \emph{conditional elimination}
step in Kapur's algorithm. We notice that this 
step \emph{propagates equationally} the
head equations of grounded Horn clauses with 
common antecedents. Initially we employ the
unsatisfiability algorithm for Horn clauses 
to achieve such propagation. However, the original
algorithm will not be enough because it will 
only propagate the head equation when all the
antecedents have truth value equal to 
true. To fix that problem, we modify two steps in Gallier's
algorithms:

\begin{itemize}
  \item When we build the data structure \emph{numargs} that keeps 
    track of the number of unproven
    equations in the antecedent of each Horn clause, we change 
    this number by the number
    of unproven uncommon equations in the antecedent of each 
    Horn clause. This will be useful
    because we only introduce head equations intro the queue 
    data structure in Gallier's algorithm
    when all the antecedents are true. With this 
    modification, our algorithm introduces head equations
    when all the antecedent equations are common. 
    Additionally the algorithm can still update
    correctly the truth value of common equations, 
    but these are not relevant for our propagation
    purposes.
  \item To guarantee that \emph{numargs} keeps the 
    right number of uncommon equations yet to
    be proven, we also modify the update mechanism for 
    \emph{numargs} in the main while loop of the algorithm.
    The original algorithm reduces by one the 
    corresponding entry in \emph{numargs}
    whenever a recently popped element from the queue 
    matches the antecedent of a Horn clause. We only
    decrease this value if such popped equation is uncommon. 
    This prevents the algorithm from accidentally
    reducing the number of uncommon equations yet to be proven, 
    which can cause that we propagate the
    uncommon head equation when the antecedent of a Horn 
    clause only consists of common equations.
\end{itemize}

At the end of this algorithm we can identify 
\emph{usable Horn clauses} by checking the Horn clauses
with \emph{numargs} entries equal to 0. Nonetheless, these 
Horn clauses are not the
desired \emph{usable Horn clauses} because the 
unsatisfiability testing algorithm
did not update the antecedents of the Horn clauses. 
The main difficulty to design a data structure
for the latter to work inside the unsatisfiability 
testing algorithm was the queue data structure
only adds grounded equation whenever the truth 
value of the literal changes to true, which happens
during \emph{equational propagation} or during 
the \emph{implicational propagation} steps.
For the \emph{implicational propagation} the task is 
easy because we can know the clause
where the just new proven ground equation comes, 
but it cannot be the same situation
for the \emph{equational propagation} since this 
step relies on congruence closure.

To remedy this issue, we equip our congruence closure 
algorithm with the Explanation operator, so
we can recover the grounded equations needed to entail 
any particular grounded equation. Additionally,
this will require a data structure to maintain the Horn 
clauses for each grounded equation that
it is the head equation of. With the latter we can 
recover the Horn Clauses where each grounded
equation came from to update the antecedents and 
obtain \emph{usable Horn clauses}.

The algorithm appears below in pseudo-code notation:

\begin{algorithm}
  \caption{Modified Unsatisfiability Testing for Ground Horn Clauses}
  \linespread{\separationline}\selectfont
  \begin{algorithmic}[1]
    \Procedure {satisfiable}{var H : Hornclause; var queue, combine: queuetype; 
    var GT(H) : Graph; var consistent : boolean}
    \While {queue not empty and consistent}
    \State node := pop(queue);
    \For {clause1 in H[node].clauselist}
    \If {$\neg$ clause1.isCommon()}
    \State {numargs[clause1] := numargs[clause1] - 1}
    \EndIf
    \If {numargs[clause1] = 0}
    \State nextnode := poslitlist[clause1];
    \If {$\neg$ H[nextnode].val}
    \If {nextnode $\neq \bot$ }
    \State {queue := push(nextnode, queue);}
    \State {H[nextnode].val := true;}
    \State {u := left(H[nextnode].atom);} 
    \State {v := right(H[nextnode].atom);}
    \If {FIND(R, u) $\neq$ FIND(R, v)}
    \State {combine := push((u, v), combine);}
    \EndIf
    \Else
    \State {consistent := false;}
    \EndIf
    \EndIf
    \EndIf
    \EndFor
    \If {queue is empty and consistent}
    \State {closure(combine, queue, R);}
    \EndIf
    \EndWhile
    \EndProcedure
    \Statex
    \Procedure {closure}{var combine, queue : queuetype; 
    var R : partition}
    \While {combine is not empty}
    \State (u, v) = pop(combine)
    \State MERGE(R, u, v, queue)
    \EndWhile
    \EndProcedure
  \end{algorithmic}
\end{algorithm}

\begin{algorithm}
  \caption{Modified Congruence Closure with Explanation Algorithms - Merge}
  \linespread{\separationline}\selectfont
  \begin{algorithmic}[2]
    \Procedure {Merge}{R : partition, u, v : node; queue, combine : queuetype}
    \If {u and v are constants a and b}
    \State {add a = b to Pending;} 
    \State {Propagate();}
    \Else \Comment {u=v is of the form apply(a1, a2)=a}
    \If {Lookup(Representative(a1), Representative(a2)) is some apply(b1, b2)=b}
    \State {add (apply(a1, a2)=a, apply(b1, b2) = b) to Pending;} 
    \State {Propagate();}
    \Else
    \State {set Lookup(Representative(a1), Representative(a2)) to apply(a1, a2)=a;}
    \State {add apply(a1, a2)=a to UseList(Representative(a1)) and to UseList(Representative(a2));}
    \EndIf
    \EndIf
    \EndProcedure
  \end{algorithmic}
\end{algorithm}

\begin{algorithm}
  \caption{Modified Congruence Closure with Explanation Algorithms - Propagate}
  \linespread{\separationline}\selectfont
  \begin{algorithmic}[2]
    \Procedure {Propagate} {\text{ }}
    \While {Pending is non-empty}
    \State {Remove E of the form a=b or (apply(a1, a2) = a, apply(b1, b2) = b) from Pending}
    \If {$Representative(a) \neq Representative(b)$ and w.l.o.g. $|ClassList(Representative(a))| \leq |ClassList(Representative(b))|$}
    \State {oldReprA := Representative(a);}
    \State {Insert edge $a \rightarrow b$ labelled with E into the proof forest;}
    \For {each c in ClassList(oldReprA)}
    \State {set Representative(c) to Representative(b)}
    \State {move c from ClassList(oldReprA) to ClassList(Representative(b))}
    \For {each pointer L in ClassList(u)}
    \If {H[L].val = false}
    \State {set the field H[L].lclass or H[L].rclass pointed to by p to Representative(b)}
    \If {H[L].lclass = H[L].rclass}
    \State {queue := push(L, queue);}
    \State {H[L].val := true}
    \EndIf
    \EndIf
    \EndFor
    \EndFor
    \For {each apply(c1, c2) = c in UseList(oldReprA)}
    \If {Lookup(Representative(c1), Representative(c2)) is some apply(d1, d2) = d}
    \State {add (apply(c1, c2) = c, apply(d1, d2) = d) to Pending;}
    \State {remove apply(c1, c2) = c from UseList(oldReprA);}
    \Else
    \State {set Lookup(Representative(c1), Representative(c2)) to apply(c1, c2) = c;}
    \State {move apply(c1, c2) = c from UseList(oldReprA) to UseList(Representative(b));}
    \EndIf
    \EndFor 
    \EndIf
    \EndWhile
    \EndProcedure
  \end{algorithmic}
\end{algorithm}

\subsection{Ground Horn Clauses with Explanations}

We notice that, by removing our changes to the unsatisfiability testing
for grounded Horn clauses regarding uncommon symbols, we effectively combine
the congruence closure with explanations to the original unsatisfiability
testing algorithm. With the latter, we can query the membership of a Horn
clauses in a given user-defined theory and additionally obtain a proof of
the latter. This approach works by introducing the antecedent equations of
a grounded Horn clause as part of the user-defined theory in order to prove
its head equation. By the Deduction Theorem \cite{10.5555/1642730}, we can
recover a proof of the original queried Horn clause by removing the antecedent
equations appearing the proof given by the Explain operation.

\subsection{New optimized conditional replacement step in Kapur's algorithm}

Once the conditional congruence closure data structure is built after 
the execution of the previous step, we can compute conditional eliminations
as follows. For the latter, we will require the following auxiliary functions:

\begin{algorithm}
  \caption{Auxiliary function - Candidates}
  \linespread{\separationline}\selectfont
  \begin{algorithmic}[2]
    \Procedure {Candidates} {z3::expr const \& t}
    \If {t is common}
    \State {\textbf{return} $\{ t \}$}
    \Else
    \State {\textbf{return} $\{ t^{'} | t^{'}  \in Class(t), t^{'} \text{is common} \}$}
    \EndIf
    \EndProcedure
  \end{algorithmic}
\end{algorithm}

\begin{algorithm}
  \caption{Auxiliary function - Auxiliar explain}
  \linespread{\separationline}\selectfont
  \begin{algorithmic}[2]
    \Procedure {Explain} {z3::expr const \& t1, z3::expr const \& t2}
    \State {z3::expr\_vector ans;}
    \If {t1.id() equals t2.id()}
    \State {\textbf{return} ans;}
    \EndIf
    \State {auto partial\_explain = hsat.equiv\_class.explain(t1, t2);} 
    \For {(auto const \& element : partial\_explain)}
    \If {element is common}
    \State {ans.push\_back(element);}
    \Else
    \State {auto const \& entry = hsat.head\_term\_indexer.find(equation.id());}
    \If {entry equals hsat.head\_term\_indexer.end()}
    \If {equation is common}
    \State {ans.push\_back(equation)};
    \EndIf
    \Else
    \For {(auto const \& hsat\_equation : entry $\rightarrow$ second $\rightarrow$ getAntecedent())}
    \State {ans.push\_back(hsat\_equation);}
    \EndFor
    \EndIf
    \EndIf
    \EndFor
    \EndProcedure
  \end{algorithmic}
\end{algorithm}

\begin{algorithm}
  \caption{Auxiliary function - allCandidates}
  \linespread{\separationline}\selectfont
  \begin{algorithmic}[2]
    \Procedure {allCandidates} {z3::expr const \& t}
    \If {t has f-symbol uncommon}
    \State {\textbf{return} $\{\{\}\}$;}
    \EndIf
    \If {t has f-symbol common and is of the form $f(t_1, \dots, t_n)$}
    \State {\textbf{return} $\{candidates(t_1), \dots, candidates(t_n)\}$;}
    \EndIf
    \If {t is a constant}
    \State {undefined}
    \EndIf
    \EndProcedure
  \end{algorithmic}
\end{algorithm}

\begin{algorithm}
  \caption{Conditional Elimination - Part 1}
  \linespread{\separationline}\selectfont
  \begin{algorithmic}[2]
    \Procedure {Conditional Elimination} {z3::expr const \& x, z3::expr const \& y}

    \If {x is constant and y is constant}
    \For {$\sigma_x$ in CANDIDATES(x)}
    \For {$\sigma_y$ in CANDIDATES(y)}
    \State {horn\_clause.add(EXPLAIN(x, $\sigma_x$) + EXPLAIN(y, $\sigma_y$), $\sigma_x = \sigma_y$)}
    \EndFor
    \EndFor
    \EndIf

    \If {x is constant and y is of the form $f_y(t_1^{'}, \dots, t_{k_2}^{'})$}
    \For {$\sigma_x$ in CANDIDATES(x)}
    \For {$\sigma_{f_y}$ in CANDIDATES($f_y(t_1^{'}, \dots, t_{k_2}^{'})$)}
    \State {horn\_clause.add(EXPLAIN(x, $\sigma_x$) + EXPLAIN($f_y(t_1^{'}, \dots, t_{k_2}^{'})$), $\sigma_y$), $\sigma_x = \sigma_{f_y}$}
    \EndFor
    \For {$arguments_{f_y}$ in CARTESIANPROD(ALLCANDIDATES($f_y(t_1^{'}, \dots, t_{k_2}^{'})$))}
    \State {horn\_clause.add(EXPLAIN(x, $\sigma_x$) + $\sum_{i=1}^{k_2}$ EXPLAIN($t_i^{'}$, $arguments_{f_y}[i]$), $\sigma_x = f_y(arguments_{f_y})$)}
    \EndFor
    \EndFor
    \EndIf

    \If {x is of the form $f_x(t_1, \dots, t_{k_1})$ and y is a constant}
    \State {\textbf{return} CONDITIONAL ELIMINATION(y, x);}
    \EndIf

    \EndProcedure
  \end{algorithmic}
\end{algorithm}



\begin{algorithm}
  \caption{Conditional Elimination - Part 2}
  \linespread{\separationline}\selectfont
  \begin{algorithmic}[2]
    \Procedure {Conditional Elimination} {z3::expr const \& x, z3::expr const \& y}

    \If {x is of the form $f_x(t_1, \dots, t_{k_1})$ and y is of the form $f_y(t_1^{'}, \dots, t_{k_2}^{'})$}
    \For {$\sigma_{f_x}$ in CANDIDATES($f_x(t_1, \dots, t_{k_1})$)}
    \For {$\sigma_{f_y}$ in CANDIDATES($f_y(t_1^{'}, \dots, t_{k_2}^{'})$)}
    \State {horn\_clause.add(EXPLAIN($f_x(t_1, \dots, t_{k_1})$, $\sigma_{f_x}$) + EXPLAIN($f_y(t_1^{'}, \dots, t_{k_2}^{'})$), $\sigma_y$), $\sigma_{f_x} = \sigma_{f_y}$}
    \EndFor
    \For {$arguments_{f_y}$ in CARTESIANPROD(ALLCANDIDATES($f_y(t_1^{'}, \dots, t_{k_2}^{'})$))}
    \State {horn\_clause.add(EXPLAIN($f_x(t_1, \dots, t_{k_1})$, $\sigma_{f_x}$) + $\sum_{i=1}^{k_2}$ EXPLAIN($t_i^{'}$, $arguments_{f_y}[i]$), $\sigma_{f_x} = f_y(arguments_{f_y})$)}
    \EndFor
    \EndFor

    \For {$arguments_{f_x}$ in CARTESIANPROD(ALLCANDIDATES($f_x(t_1, \dots, t_{k_1})$))}
    \For {$\sigma_{f_y}$ in CANDIDATES($f_y(t_1^{'}, \dots, t_{k_2}^{'})$)}
    \State {horn\_clause.add($\sum_{i=1}^{k_1}$ EXPLAIN($t_i$, $arguments_{f_x}[i]$) + EXPLAIN($f_y(t_1^{'}, \dots, t_{k_2}^{'})$), $\sigma_y$), $f_x(arguments_{f_x}) = \sigma_{f_y}$}
    \EndFor
    \For {$arguments_{f_y}$ in CARTESIANPROD(ALLCANDIDATES($f_y(t_1^{'}, \dots, t_{k_2}^{'})$))}
    \State {horn\_clause.add($\sum_{i=1}^{k_1}$ EXPLAIN($t_i$, $arguments_{f_x}[i]$) + $\sum_{i=1}^{k_2}$ EXPLAIN($t_i^{'}$, $arguments_{f_y}[i]$), $f_x(arguments_{f_x}) = f_y(arguments_{f_y})$)}
    \EndFor
    \EndFor

    \EndIf

    \EndProcedure
  \end{algorithmic}
\end{algorithm}

TODO: add more information.

%%% Local Variables:
%%% mode: latex
%%% TeX-master: "main"
%%% End:
