This theory appears heavily in formal methods dealing
with abstract domains, introduced in 
\cite{journals/corr/abs-cs-0703084}, which have been useful
for efficient linear invariant discovery, safety analysis, and
static analysis of programs.
The decision problem consists of checking the satisfiability
of a particular fragment of $LIA(\mathbb{Z})$. The fragment 
consists on conjunctions of inequalities with at most
two variables which integers coefficients are restricted
to $\{-1, 0, 1\}$. Efficient algorithms are found in the literature 
for both the satisfiability problem \cite{10.1007/11559306_9} and as 
well as for interpolation 
\cite{10.1007/978-3-642-02959-2_15} of the theory. 

The algorithm in \cite{KAPUR2017} follows a similar approach to the
uniform interpolation algorithm for EUF in the sense 
that the attention is given to one of 
the formulas in the interpolation pair \footnote {This implementation uses the first formula of the pair. }.
Other approaches towards interpolation follow graph-based 
algorithms which idea combines the reduction of 
UTVPI formulas to difference 
logic \cite{journals/corr/abs-cs-0703084} and a cycle 
detection of maximal size \cite{10.1007/978-3-642-02959-2_15}.

%%% Local Variables:
%%% mode: latex
%%% TeX-master: "main"
%%% End:
