\section{A partially sound 
  uniform interpolant generation algorithm 
  for the combined theory of EUF and UTVPI}

This section introduces a partially sound uniform interpolant
generation algorithm for the aforementioned theories in the
sense that it produces a uniform interpolant if for
all variables $x$ to eliminate in the UTVPI components
of the input formula $\psi$, either:
\begin{itemize} \label{weakening_conditions}
  \item $\psi \models_{EUF + UTVPI} x \leq n_1$
    and $\psi \models_{EUF + UTVPI} -x \leq n_2$ where
    $n_1, n_2 \in \mathbb{Z}$, or
  \item There exists $a_1 x + a_2 y$ with $y$ a common
    variable such that \\
    $\psi \models_{EUF + UTVPI} a_1 x + a_2 y \leq n_1$ and
    $\psi \models_{EUF + UTVPI} -a_1 x - a_2 y \leq n_2$, 
    where $a_1, a_2 \in \{-1, 0, 1\}$ and 
    $n_1, n_2 \in \mathbb{Z}$
\end{itemize}

The algorithm is a tableux-like algorithm which specifies
formula state and reduction rules.
It extends the approach in \cite{ghilardi2020compactly}
by incorporating additional structure to handle 
UTVPI formulas and more reduction rules to handle
the Normalize and Elim rules introduced in the
uniform interpolant generation algorithm for
UTVPI theory in Chapter 4. 

\subsection{A tableux-like uniform interpolant 
  generation algorithm for EUF}

In this section, we review 
the uniform interpolant generation algorithm
for EUF from \cite{ghilardi2020compactly}. The algorithm 
follows a
similar preprocessing step as the uniform interpolation 
algorithm for EUF 
discussed in Chapter 3. The `formula state'
encodes a triplet $\langle \delta(\cev{y}, \cev{z}), 
\phi(\cev{y}, \cev{z}), \psi(\cev{e}, \cev{y}, \cev{z})
\rangle$ of the resulting formula after preprocessing 
with the following meaning:

\begin{itemize}
  \item $\cev{e}$ encodes information about 
    the variables to be eliminated.
  \item $\cev{y}$ encodes information about 
    the variables to be eliminated
    than became common variables since the 
    algorithm detected an equality
    of a variable in this category with a common term.
  \item $\cev{z}$ encodes information about the 
    common variables. It is worth
    mentioning that each \cev{e}, \cev{y}, and \cev{z} 
    are indexed terms
    which encode an ordering relation between them.
  \item $\delta(\cev{y}, \cev{z})$ stands for the formulas 
    that provide
    an explicit definition for variables in 
    $\cev{y}$ using a DAG-representation.
  \item $\phi(\cev{y}, \cev{z})$ stands for the formulas 
    that do not contain
    variables to eliminate.
  \item $\psi(\cev{e}, \cev{y}, \cev{z})$ stands 
    for the formulas
    that contain variables to eliminate.
\end{itemize}

Some definitions are needed before discussing 
the tableux rules in 
\cite{ghilardi2020compactly}.

\begin{definition}
  A term $t$ (resp. literal $L$) is $\cev{e}$-free 
  when there is no occurrence
  of variables to eliminate in $t$ (resp. $L$).

  Two flat terms $t, u$ of the form 
  \begin{equation*}
    t := f(a_1, \dots, a_n), u := f(b_1, \dots, b_n)
  \end{equation*}
  are said to be \emph{compatible} if and only if every $i = 1, \dots, n$, either
  $a_i$ is identical to $b_i$ or both $a_i$ and $b_i$ are $\cev{e}$-free.

  The \emph{difference set of two compatible terms} like 
  above is the
  set of dis-equalities $a_i \neq b_i$ such that 
  $a_i$ is not identical
  to $b_i$.
\end{definition}

The algorithm in \cite{ghilardi2020compactly} provides 
the following tableux rules:

\begin{itemize}
  \item[] 1. Simplification rules:
    \begin{itemize}
      \item 1.0) If an atom like $t = t$ belong to $\psi$, just remove it; if a
      literal like $t \neq t$ occurs somewhere, delete $\psi$, replace $\phi$
      with $\bot$ and stop.
    \item 1.i) If $t$ is not a variable to eliminate and $\psi$ contains
    both $t = a$ and $t = b$, remove the latter and replace it with $b = a$.
  \item 1.ii) If $\psi$ contains $e_i = e_j$ with $i > j$, remove it
  and replace every $e_i$ in $\delta, \phi, \psi$ by $e_j$.
    \end{itemize}
  \item[] 2. DAG update rule: if $\psi$ contains $e_i = t(\cev{y}, \cev{z})$, remove
    it, rename every $e_i$ in $\delta, \phi, \psi$ as $y_j$ (for fresh $y_j$)
    and add $y_j = t(\cev{y}, \cev{z})$ to $\delta(\cev{y}, \cev{z})$.
  \item[] 3.  e-Free Literal rule: if $\psi$ contains a literal $L(\cev{y}, \cev{z})$
    move it to $\phi(\cev{y}, \cev{z})$.
  \item[] 4.  If $\psi$ contains a pair of atoms $t = a$ and $u = b$, where
    $t$ and $u$ are compatible flat terms, and not dis-equality from the difference
    set of $t, u$ belongs to $\phi$, then non-deterministically apply one of the
    following alternatives:
    \begin{itemize}
      \item 4.0) Remove from $\psi$ the atom $f(b_1, \dots, b_n) = b$, add
      to $\psi$ the atom $b = a$ and add to $\phi$ all the equalities
      $a_i = b_i$ such that $a_i \neq b_i$ is in the difference set of $t, u$;
      \item 4.1) Add to $\phi$ one of the dis-equalities form the difference
      set of $t, u$.
    \end{itemize}
\end{itemize}

\subsection{Our proposed partially sound 
uniform interpolant generation algorithm 
for EUF + UTVPI}

Our algorithm first purifies a given satisfiable 
input formula in the
combined theory of EUF and UTVPI and 
uses an extended tableux-like algorithm
using additional rules to deal with UTVPI formulas. 
If the input formula is not satisfiable, 
then return $\bot$ as result. 
We use the data structures discussed 
in Chapter 4 to deal with UTVPI formulas,
hence we keep a normal form representation 
of the UTVPI inequalities in the
formula state inside the proposed 
tableux-like algorithm.
The previous rules of the uniform interpolant 
generation algorithm
for EUF involving propagation of equations 
suits the UTVPI component as well. 
The additional rules are the following:

\begin{itemize}
  \item[] 5. Eliminate uncommon UTVPI terms: 
    if there are UTVPI inequalities of the form $a_i x + a_j e_j \leq k_1$
    and $a_k y - a_j e_j \leq k_2$ in $\psi$, then introduce to $\phi$ the 
    UTVPI inequality $a_i x + a_k y \leq k_1 + k_2$.

  \item[] 6. Normalize UTVPI inequalities:
    if there is a UTVPI inequality of the form $a_i x + a_i x \leq k$ in 
    the formula state, then remove it and insert to $\psi$ the UTVPI
    inequality $a_i x \leq \floor{k/2}$

  \item[] 7.  Normalize bounds:
    if there are two UTVPI inequalities of the form $a_i x + a_j y \leq k_1$,
    $a_i x + a_j y \leq k_2$ in 
    the formula state with $\{k_1, k_2\} \in \mathbb{N}$, 
    then remove them both and insert to $\psi$ the UTVPI
    inequality $a_i x + a_j y \leq min(k_1, k_2)$

  \item[] 8.  Propagate fully bounded uncommon UTVPI inequalities: 
    if there are two UTVPI inequalities 
    in $\psi$ of the form $a_i e_i + a_j e_j \leq k_1$ and -$a_i e_i - a_j e_j \leq k_2$, 
    where $\{a_i, a_j\} \subseteq \{1, -1\}$ and $i > j$ 
    then  non-deterministically apply the following rule:
    \begin{itemize}
      \item Remove both $a_i e_i + a_j e_j \leq k_1$ and $-a_i e_i - a_j e_j \leq k_2$
        from $\psi$ and replacing every $e_i$ by $e_j = l + a_j a_i e_i$ where $
        l \in \{a_j k_2, a_j k_2 + 1, \dots, a_j k_1 - 1, a_j k_1\}$.
    \end{itemize}

  \item[] 9. Propagate fully bounded uncommon UTVPI terms: 
    if there are two UTVPI inequalities 
    in $\psi$ of the form $a_i e_i + a_j e_j \leq k_1$ and 
    -$a_i e_i - a_j e_j \leq k_2$, 
    where $\{a_i, a_j\} \subseteq \{1, -1\}$ and $i > j$ 
    then  non-deterministically apply the following rule:
    \begin{itemize}
      \item Remove both $a_i e_i + a_j e_j \leq k_1$ and $-a_i e_i - a_j e_j \leq k_2$
        from $\psi$ and replacing every $e_i$ by $e_j = l + a_j a_i e_i$ where $
        l \in \{a_j k_2, a_j k_2 + 1, \dots, a_j k_1 - 1, a_j k_1\}$.
    \end{itemize}
\end{itemize} 

Remark \label{my_remark}: Notice that if no 
transformation applies 
to the formula state, we obtain the same 
the following types of formulas for the EUF 
component according to \cite{ghilardi2020compactly} 

\begin{itemize}
  \item $\psi$ only contains dis-equalities of the kind $e_i \neq a$ and equalities of
    the kind $f(a_1, \dots, a_n) = a$. If $\psi$ contains equalities of the
    kind $f(a_1, \dots, a_n) = a$ it means that at least one $a_i$ must belong
    to $\cev{e}$. If $\psi$ contains two equalities of the form
    $f(a_1, \dots, a_n) = a$ and $f(b_1, \dots, b_n) = b$ then it means
    $f(a_1, \dots, a_n)$ and $f(b_1, \dots, b_n)$ are incompatible or $a_i \neq b_i$
    belongs to $\phi$.
\end{itemize}

Similarly, the following types of formulas for 
the UTVPI component exists in the
formula state if no transformation apply: bounded 
only by an upper bound, 
bounded only by a lower bound, and fully Bounded 
with $\pm$ infinity. The
latter follows directly from the normal 
form representation of UTVPI inequalities
induced by the data structures introduced 
in Chapter 4. Otherwise, rules
8 and 9 should have applied.

\subsection{Termination, and partial soundness of the
proposed algorithm}

\begin{lemma}
  The non-deterministic procedure presented above 
  always terminates.
\end{lemma}

\begin{proof}
  Following a similar proof as in \cite{ghilardi2020compactly}
  it is enough to show every branch of the algorithm terminates.
  The rules involving the EUF component are cover in
  \cite{ghilardi2020compactly}. For the rest of the rules
  involving UTVPI expressions, we first notice that there is a 
  $\mathcal{O}(n^2)$ bound over all the UTVPI expressions
  for UTVPI signature $S$ where $n$ is the number of distinct
  variables in $S$. The rules relax the bounds of all the
  bounds in the inequalities, which does not induce a negative
  cycle, otherwise the input formula could not be satisfiable.
\end{proof} 

Before proving the partial soundness of the tableux algorithm 
for the theory combination of EUF and UTVPI, we first state
important results relating the existence of uniform 
interpolant.

\begin{lemma} \cite{ghilardi2020compactly} \label{rewriting_model_existance}
  Let $R$ be a canonical ground rewrite system over a signature
  $\Sigma$. There is a $\Sigma$-structure $\mathcal{M}$
  such that for every pair of ground terms $t, u$ we have
  that $\mathcal{M} \models t = u$ if and only if 
  $R$-normal formal of $t$ is the same as the $R$-normal form of
  $u$. Consequently, $R$ is consistent with a set of negative 
  literals $S$ if and only if for every $t \neq u \in S$ 
  the $R$-normal forms of $t$ and $u$ are different.
\end{lemma}

\begin{lemma} \cite{10.1007/978-3-030-29436-6_9} \label{cover_extension_lemma}
  Let $T$ be a theory. 
  A formula $\psi(\cev{y})$ is a uniform interpolant
  in $T$ of $\exists \cev{e} . \phi (\cev{e}, \cev{y})$
  if and only if it satisfies the following two conditions:

  \begin{itemize}
    \item[] i) $T \models \forall \cev{y}. ((\exists \cev{e} . 
    \phi(\cev{e}, \cev{y})) \rightarrow \psi(\cev{y}))$
  \item[] ii) for every model $\mathcal{M}$ of $T$, for every
  tuple of elements $\cev{a}$ from the support of $\mathcal{M}$
  such that $\mathcal{M} \models \psi(\cev{a})$ it is 
  possible to find another model $\mathcal{N}$ of $T$
  such that $M$ embeds into $\mathcal{N}$ and $\mathcal{N}
  \models \exists \cev{e} . \psi (\cev{e}, \cev{a})$
  \end{itemize}
\end{lemma}

\begin{theorem}
  Given an input formula
  $\psi := \exists \cev{e}. \phi(\cev{e}, \cev{z})$, 
  if $\psi$ satisfies the weakening conditions in 
  \ref{weakening_conditions} and the proposed algorithm
  terminates with its branches in the formula states

  $\langle 
  \delta_1(\cev{y_1}, \cev{z}), 
  \phi_1(\cev{y_1}, \cev{z}),
  \psi_1(\cev{e_1}, \cev{y_1}, \cev{z})
  \rangle, \dots, \langle 
  \delta_k(\cev{y_k}, \cev{z}), 
  \phi_k(\cev{y_k}, \cev{z}),
  \psi_k(\cev{e_k}, \cev{y_k}, \cev{z})
  \rangle$

  Then the uniform interpolant of $\exists \cev{e} . \phi(\cev{e},
\cev{z})$ in $EUF + UTVPI$ is the DAG-unravelling \footnote{
  Unravelling means that the explicit recursive 
  substitution of the each of the $y_i$
  variables are substituted in $\phi_i(\cev{y_i}, \cev{z})$ using
  the equations in $\delta_i(\cev{y_i}, \cev{z})$.
}of the formula 

$\bigvee_{i=1}^k \exists \cev{y_i} . (\delta_i(\cev{y_i}, \cev{z}) \land \phi_i(\cev{y_i}, \cev{z}))$

\end{theorem}

\begin{proof}
  The proof follows the similar style of the proof of Theorem 
  5.1 in \cite{ghilardi2020compactly}.
  Let $\alpha$ be 
  $\bigvee_{i=1}^k \exists \cev{y_i} . (\delta_i(\cev{y_i}, \cev{z}) \land \phi_i(\cev{y_i}, \cev{z}))$.
  First, we notice that it is enough to prove the 
  two items \ref{cover_extension_lemma} to show 
  our assertion. The first item 
  follows since $\alpha$ was obtained from the rules 
  in the proposed algorithm.

  To address the second item of the lemma, given a
  model $\cal{M}$ such that $\cal{M} \models \alpha$, finding
  an isomorphic embedding $\cal{N}$ such that $\mathcal{N} \models
  \exists \cev{e} . \phi(\cev{e}, \cev{z})$ is equivalent
  , using Robinson Lemma \cite{chang2013model}, to prove that
  there is a model for $\Delta(\mathcal{M}) 
  \cup \phi 
  \cup \{ a = \mathcal{I}(a)\}_{a \in \cev{y} \cup \cev{z}}$.

  Following \cite{ghilardi2020compactly}, we can obtain 
  a model for the
  EUF component using the model induced by the canonical 
  form obtained by orienting the equations in $\psi$. 
  In other to obtain a model for the 
  UTVPI component, we noticed that if the 
  input formula satisfies the weakening 
  conditions in \ref{weakening_conditions}
  then either:
  \begin{itemize}
      \item If the uncommon variable is bounded 
        from below and 
        bounded from above, 
        then we can find a model for these variables
        by picking an integer
        in the respective range of the bounds.
      \item If the above condition does not
        hold, then there exists an UTVPI expression 
        $a_1 x + a_2 y$ such that
        $\psi \models_{EUF + UTVPI} n_2 \leq 
        a_1 x + a_2 y \leq n_1$, where $y$ is common.
        Thus, we can generate a canonical rewrite
        system for each of the branches that the
        rule 9. in the proposed algorithm by 
        adding rules of the form 
        $x \rightarrow a_1(a_2 y + i)$ where
        $i \in \{n_2, \dots, n_1\}$. A model 
        from this induced canonical rewrite system 
        can be obtained by Lemma 
        \ref{rewriting_model_existance}.
  \end{itemize}
  
  With the latter, we can find a model $\mathcal{N}$
  satisfying $\exists \cev{e} . 
  \psi(\cev{e}, \cev{y}, \cev{z})$
  with the desired conditions. 
  Therefore, $\alpha$ is an uniform interpolant
  for $\exists \cev{e}. \phi(\cev{e}, \cev{z})$.

\end{proof}

\subsection{Illustrating 
Example}

Let us consider the following input formula in the
combined theory satisfying the weakening conditions in 
\ref{weakening_conditions}:

$formula$

%%% Local Variables:
%%% mode: latex
%%% TeX-master: "main"
%%% End:


%%% Local Variables:
%%% mode: latex
%%% TeX-master: "main"
%%% End:
