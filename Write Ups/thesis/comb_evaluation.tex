\section{Evaluation}

\subsection{Detailed evaluation of a complete example}

\subsubsection{A simple example}

Let us consider the following example: 
$\alpha = \{ f(x_1) = 0, x_1 = a, y_1 \leq a\}; 
\beta = \{x_1 \leq b, y_1 = b, f(y_1) \neq 0\}$.
The implementation produces the following interpolant \footnote{A trace of the
execution can be found at the Appendix section since it is considerably
large to include it in this section}: 

\verbatiminput{th_comb_example_interp.txt}

The following SMT query verifies that the previous result obtained is 
an interpolant of the input formula:

\verbatiminput{../../Software/Cpp/ThCombination/tests/verification/verification_basic_test_actual_example.smt2}

For this example these are the interpolants reported by Z3 and Mathsat
respectively:

\begin{itemize}
\item Z3 interpolant: (and ($\geq$ (+ x1 (* (- 1) y1)) 0) (= (f x1) 0))
\item Mathsat interpolant: (or (= (f y1) 0) ($\leq$ 1 (+ x1 (* (- 1) y1))))
\end{itemize}

The following SMT query verifies the strength relation between the interpolants produced:

\verbatiminput{../../Software/Cpp/ThCombination/tests/verification/basic_test_actual_example_final_check.smt2}


\subsection{A parametrized example}

Let us consider the following parametrized 
example: $\alpha = \{1 \leq x, x \leq n \}; \beta = \{f(x) = a, 
f(1) \neq a,
f(2) \neq a, \dots
f(n-1) \neq a,
f(n) \neq a\}$ where $n > 1, a$ are fixed integers. We can see that an 
interpolant for this parametrized input problem is 
$x = 1 \lor x = 2 \lor \dots \lor x = n-1 \lor x = n$.

For the case $n = 3, 4, 5$, both the implementation, Z3, and Mathsat were able
to compute the expected result discussed above. This particular test
was useful to check disjunction of equalities propagation mechanism in the
implementation work.

\subsection{Performance comparision}

The following collection of examples were produced from a reduction 
algorithm that takes an input formula in the theory of arrays combined with
total orders, reducing the formula to EUF and total order. We can use our
implementation of theory combination of EUF + UTVPI formulas because 
given an expression of the form $a \star b$ where
$\star$ is either $<, \leq, >, \geq, =, \neq$, 
the latter is equivalent to 
simplify($a - b \star 0$) using Z3's arithmetic rewriter. 
The latter is justified formally since integers 
satisfy the cancellation property over addition.


TODO: keep working here

\begin{tabular}{llcccccc} \toprule
  \multicolumn{2}{l}{Input Formula} & \multicolumn{2}{c}{Z3}& \multicolumn{2}{c}{Mathsat}& \multicolumn{2}{c}{The implementation} 
  \\\cmidrule(lr){1-2}\cmidrule(lr){3-4}\cmidrule(lr){5-6}\cmidrule(lr){7-8}
  A-part & B-part & Result & Time (m.s.) & Result & Time (m.s.) & Result & Time (m.s.) \\
  
  %&  & & & & & & \\
\end{tabular}

%%% Local Variables:
%%% mode: latex
%%% TeX-master: "main"
%%% End:
