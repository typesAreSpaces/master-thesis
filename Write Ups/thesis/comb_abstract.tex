Theory combination techniques involve 
reusing the algorithms for verification 
problems of some of the theories involved 
by either purifying terms over the 
theories, or reducing the original problem 
into a base theory, or a combination of these two. 
Usually the approaches following the first approach 
aforementioned rely on a Nelson-Oppen
framework \cite{10.1007/11532231_26, 
10.1007/978-3-642-22119-4_1, 10.1145/2490253}

Despite possibly more efficient approaches using
model-based theory combination techniques 
\footnote{
  An interesting property of model-based theory combination 
  is that it is not needed to propagate disjunctions, even if
  the theories are not convex. The reason for this is because
  the \emph{disjunctiviness} of the problem is handled 
  \emph{lazily} by the
  SAT solver.
} \cite{10.1007/978-3-642-22119-4_1}, the latter requires 
operations which many SMT solvers do not provide 
proper API to implement these.
Hence, the approach used in the thesis work follows 
\cite{10.1007/11532231_26} since it does not require extensive
modification to the decision procedures
used. Many of the necessary modifications 
were implemented on top of Z3 and PicoSAT/TraceCheck.

The propagation of new equalities and disjunction of equalities
requires the additional step to split these formulas into
the correct part of the interpolant pair. Among the major
contributions of \cite{10.1007/11532231_26} was the introduction
of the class of equality interpolanting theories.

\begin{definition}
  A theory $\cal{T}$ is \emph{equality interpolanting}
  if for every $A$, $B$ in $\cal{T}$ and every AB-mixed
  equality $a = b$ such that $A \land B \models_{\cal{T}} a = b$,
  there exits a term $t$
  in $\cal{T}$ (called interpolanting term)
  such that $A \land B \models_{\cal{T}} a = t$ and 
  $A \land B \models_{\cal{T}} b = t$.
\end{definition}

The relevance of the existance of the interpolating term for 
a deduced AB-mixed equality becomes relevant in the context 
of splitting a formula into a suitable A-part and B-part 
respectively. If t

Deciding where to include AB-common terms to either the 
A-part or the B-part of the interpolantion pair affects 
the final result since the interpolant will be \emph{closer} 
to the A-part or to the B-part respectively. The authors in 
\cite{10.1007/11532231_26} include AB-common terms to the B-part. 
However, the implementation work includes AB-common terms to the
A-part due to the interest in uniform interpolants.

%%% Local Variables:
%%% mode: latex
%%% TeX-master: "main"
%%% End:
