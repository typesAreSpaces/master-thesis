\chapter{Interpolation algorithm for the theory of EUF}

Interpolation algorithms for the theory of equality 
with uninterpreted functions are relevant as the core 
component of verification algorithms. Many useful techniques 
in software engineering like bounded/unbounded model 
checking and invariant generation benefit directly from 
this technique. In \cite{10.1007/3-540-58179-0_44}, the 
authors introduced a methodology to debug/verify the 
control logic of pipelined microprocessors by encoding 
its specification and a logical formula denoting the 
implementation of the circuit into a EUF solver.

Previous work addressing the interpolation problem for 
EUF has involved techniques ranging from
interpolant-extraction from refutation proof 
trees \cite{10.1007/978-3-540-24730-2_2, mcmillan2011interpolants, 
10.1007/978-3-642-31612-8_24}, and colored congruence closure
graphs \cite{10.1007/978-3-642-00768-2_34}. Kapur's algorithm 
uses a different approach by using approximated
quantifier-elimination, a procedure that given a formula, 
it produces a logically equivalent formula
 without a variable in particular \cite{DBLP:books/daglib/0076838}.

\section{Algorithm}

\section{Kapur's Uniform Interpolation Generation Algorithm for EUF}

Kapur's interpolation algorithm for the EUF theory uses quantifier-elimination
techniques to remove symbols in the first formula of an inconsistent pair of formulas
that are not common with the second formula of the latter. 
Hence, the input for this algorithm is a conjunction of equalities in the
EUF theory and a set of symbols to eliminate, also known as uncommon symbols.
In preparation to discuss Kapur's algorithm we need to provide the 
following definitions.

\begin{definition} 
  Let $f$ be an $n-ary$ function symbol and $a_1, \dots, a_n,
  b$ terms from the EUF language. We say 
  \begin{equation*}
    f(a_1, \dots, a_n) = b
  \end{equation*}
  is an $f-equation$ if the terms $a_1, \dots, a_n, b$ are constants in
  the EUF language. We refer to the \emph{outermost symbol of the f-equation}
  as the function symbol appearing in such $f-equation$.
\end{definition}

$f-equations$ are used in Kapur's algorithm to simplify the structure of terms 
and in Phase II to expose hidden arguments and eliminate uncommon function symbols.

As part of the input of Kapur's algorithm, there is a set of uncommon symbols given 
directly or computed from the inconsistent pair of formulas by inspection. Using
these symbols we can define the following recursive definition of \emph{uncommon terms}:

\begin{definition}
  A term $t$ in the EUF language is \emph{uncommon} if:
  \begin{itemize}
    \item $t$ is an uncommon constant
    \item $t$ is a function application of the form $f(t_1, \dots, t_n)$
      where either $f$ is an uncommon symbol or any $t_i$ where $1 \leq i \leq n$ 
      is an uncommon term
  \end{itemize}

  Similarly, we can extend homomorphically the notion of uncommon terms to uncommon 
  predicates following a similar construction.

  \begin{itemize}
    \item Let $t_1, t_2$ be terms in the EUF language. $t_1 = t_2$,
      $t_1 \neq t_2$ are uncommon predicates if either $t_1$ or $t_2$
      are uncommon terms.
    \item Let $\psi, \varphi$ be predicates in the EUF language. 
      $\psi \star \varphi$ are uncommon predicates if either $\psi$
      or $\varphi$ are uncommon predicates where $\star \in \{ \land,
      \lor, \rightarrow \}$. $\neg \psi$ is an uncommon predicate if $\psi$
      is an uncommon predicate.

  \end{itemize}
\end{definition}


Regarding extensions of the language with new constants, the \emph{uncommon} property
is preserved under equalities. Formally, we mean the following:

\begin{definition}
  Let $\mathfrak{L}$ be a EUF language and $\mathfrak{a}$ a constant symbol
  not belonging to $\mathfrak{L}$. We say $\mathfrak{a}$ is a \emph{common constant under
  the theory $\mathcal{T}$ in the extended language $\mathfrak{L} \cup \{\mathfrak{a}\}$}
  if there exists a common term $t$ in the language $\mathfrak{L}$ such that 
  $\models_{\mathcal{T}} t = \mathfrak{a}$, otherwise $\mathfrak{a}$ is an \emph{
  uncommon constant}.
\end{definition}

Since congruence closure algorithms are relevant to Kapur's algorithm, we introduce the 
following definition for notation purposes:

\begin{definition}
  Let $\mathcal{E}$ be an equivalence relation between grounded terms of some 
  language $\mathfrak{A}$. The function $repr_{\mathcal{E}} : \mathfrak{A} \rightarrow 
  \mathfrak{A} , repr_{\mathcal{E}} : a \mapsto b$  where $b$ is the representative
  element in $\mathcal{E}$ for $a$.

\end{definition}

The interpolating formula produced by Kapur's algorithm is a conjunction of equations
and Horn clauses. A Horn clause if a disjunction of literals which contain at most
one non-negated literal.
Due to its relevance in the procedure in the EUF theory, for a Horn clause $h$
we denote $antecedent(h)$ to be the conjunction of disequations in the disjunction of $h$ 
and $head(h)$ to be either the equation in the disjunction if such is present in $h$ 
or the particle $\bot$ otherwise.

The main steps in Kapur's algorithm for interpolant generation for the EUF theory
are the following:

\begin{itemize}
  \item \textbf{Flattening:} 
    For each sub-term $t$ in the input formula assign a fresh unique constant $\mathfrak{a}_t$. 
    Additionally, for each sub-term $t$ generate new equations of the form:
    \begin{itemize}
      \item $c = \mathfrak{a}_c$, if $t$ is a constant $c$
      \item $f(\mathfrak{a}_{t_1}, \dots, \mathfrak{a}_{t_n}) 
        = \mathfrak{a}_{f(t_1, \dots, t_n)}$, if $t$ is a 
        function application of the form $f(t_1, \dots, t_n)$
    \end{itemize}
    Clearly, we can see that this step generates $f-equations$.
  \item \textbf{Elimination of uncommon terms using congruence closure:}
    This step builds an equivalence 
    relation $\mathcal{E}$ of the $f-equations$ introduced in the 
    Flattening step using a congruence closure algorithm such 
    that the representatives are common terms
    whenever possible. Uncommon terms appearing in the current conjunction of 
    equations are replaced by their representatives.
  \item \textbf{Horn clause generation by exposure:} 
    This step produces for all pairs of $f-equations$ 
    $(f(\mathfrak{a}_1, \dots, \mathfrak{a}_n) = \mathfrak{c}, 
    f(\mathfrak{b}_1, \dots, \mathfrak{b}_n) = \mathfrak{d})$
    Horn clauses of the form 
    $\bigwedge_{i=1}^n(repr_{\mathcal{E}}(\mathfrak{a}_i) = repr_{\mathcal{E}}(\mathfrak{b}_i))
    \rightarrow repr_{\mathcal{E}}(\mathfrak{c}) = repr_{\mathcal{E}}(\mathfrak{d})$
    when any of the two following situations happen \footnote{Trivial equations in the antecedent
    of a Horn clause are removed; if the head equation of Horn clause produced in this 
    step is trivial then such Horn clause is discarded}:
    \begin{itemize}
      \item The outermost symbol of the $f-equations$ is an uncommon symbol.
      \item There is at least one constant argument in any of the $f-equations$ 
        that is an uncommon constant.
    \end{itemize}
  \item \textbf{Conditional elimination:} 
    We identify the Horn clauses $h := \bigwedge_i (c_i = d_i) \rightarrow a = b$
    that have \emph{common antecedents} and uncommon head equations. 
    We perform the following procedure: 

    \begin{itemize}
      \item if $a$ and $b$ are both uncommon terms:
        replace the equation $a = b$ appearing in the antecedents
        of all the current Horn clauses by $antecedent(h)$.
      \item if either $a$ is common and $b$ uncommon: replace $b$ by $a$
        in all the current Horn clauses $h^{'}$ and append $antecedent(h)$ to 
        $antecedent(h^{'})$.
      \item if either $a$ is uncommon and $b$ common: Proceed similarly as in the
        previous case.
    \end{itemize}

    We repeat this step until we cannot produce any new Horn clauses. 

  \item \textbf{Conditional replacement:} 
    For each Horn clause of the form $\bigwedge_i(a_i = b_i) 
    \rightarrow u = c$ 
    where the antecedent is common, the term $u$ in its 
    head equation is an uncommon term,
    and the term $c$ is a common term, 
    replace every instance of $u$ appearing in each 
    $f-equation$ by $c$ to generate Horn clauses with 
    antecedent $\bigwedge_i a_i = b_i$.

    Return the conjunction of formulas obtained 
    as the interpolant.
\end{itemize}

If the user is not interested in an explicit interpolant, 
we can present a \textbf{lazy/pseudo interpolant}
which is an ordered sequence of the original equations 
together with the Horn clauses produced in Phase II 
using an appropriate order between the uncommon terms. In order to
provide a description of a procedure for the latter we need
to introduce the following definitions:

\begin{definition}\label{dep_pair}
  Let $\succ$ be a partial order between terms 
  such that $a \succ b$ whenever 
  $a$ is common and $b$ is uncommon. A \emph{dependency pair}
  for a horn clause $h := \bigwedge_i (a_i = b_i) \rightarrow c = d$
  is a pair $(min(c, d, \succ), \{max(c, d, \succ)\} \cup \{ u | u$ 
  is an uncommon term appearing in antecedent(h) $\})$.
  The first element of a dependency pair is denoted as the
  target of $h$ and the second element the source of $h$.

  A \emph{valid dependency pair} is a dependency pair for some
  Horn clause $h$ which its target is not included in its
  source.
\end{definition}

We can notice from definition \ref{dep_pair} than even for 
equations, its source is never empty. 

\begin{definition}\label{dep_graph}
  Let $\succ$ be a partial order between terms 
  such that $a \succ b$ whenever 
  $a$ is common and $b$ is uncommon. Given a set of 
  Horn clauses $H$, a \emph{dependency graph for H}
  is the directed graph $G_H = (V, E)$ where
  \begin{itemize}
    \item $V := \{(target_i, source_i) | (target_i, source_i)$ 
      is a valid dependency pair from $H \}$
    \item $E := \{(target_i, source_i) \rightarrow (target_j, source_j) | 
        \{target_i\} \cup source_i \subseteq source_j
      \}$ 
  \end{itemize}
\end{definition}

Using \emph{valid dependency pairs} of the Horn clauses 
produced in Phase II
we can construct an acyclic directed graph as shown 
by the following theorem:

\begin{theorem}
  For any set of Horn clauses $H$, its dependency graph 
  never contains a cycle between its nodes.
\end{theorem}

\begin{proof}
  Suppose there exists a sequence of $n$ nodes such that 
  $(target_1, source_1) \rightarrow \dots 
  \rightarrow (target_n, source_n) \rightarrow (target_1, source_1)$.
  Since $\subseteq$ is transitive we can conclude that 
  $target_1 \in source_1$, which leads to a contradiction since all
  the nodes in a dependency graph are valid dependency pairs.
\end{proof}

Thus, given a set of Horn clauses $H$, we can compute its
dependency graph and use a topological sort algorithm to 
produce the ordered sequence required in the lazy interpolant
representation. Lazy interpolants avoid the 
possible exponential size of the 
formal interpolant. This representation 
is useful because it provides a more compact
representation of the interpolant that the user 
might be able quicker to obtain.
Additionally, the user might be just 
interested in a particular sub-formula of 
the interpolant, so the
latter representation offers such feature. This algorithm 
allows a flexible implementation 
which can lead several optimizations
based on the nature and applications of the 
interpolant.

In order to efficiently work with Horn clauses 
during and after the Conditional elimination step
in Kapur's algorithm, \cite{KAPUR2019} introduced 
\emph{conditional congruence closure} as 
an extension of the congruence closure
generated by a conjunction of equalities.
This structure includes Horn clauses in the set
of consequences of the theory induced by
the input formulas, allowing the membership checking 
of Horn clauses as well. 

\begin{definition}
  Let $S$ be a set of equations in the EUF language
  $\mathfrak{L}$, and
  $CC(S)$ the set of consequences of $S$ using congruence
  closure. Then the \emph{conditional congruence closure of S},
  abbreviated as $CCC(S)$,
  is defined as follows:
  \begin{equation*}
    H \rightarrow a = b \in CCC(S) \text{ if and only if } 
    a = b \in CC(S \cup H)
  \end{equation*}

  where $H$ is a conjunction of equations and $a,b$ terms in 
  $\mathfrak{L}$.
\end{definition}

%%% Local Variables:
%%% mode: latex
%%% TeX-master: "main"
%%% End:


%\begin{theorem}
  Let $t$ be an uncommon term and let $H$ be a collection of Horn equations.
  Assume $antecedent(H) \neq \{\}$. If $consequent(t) = \{\}$, then we cannot
  conditionality eliminate $t$ from any Horn equation $h \in H$. 
\end{theorem}

\begin{proof}
  Suppose, by contradiction, that we can conditionally eliminate the term
  $t$ from $H$. Then, there exists $h^{'} \in H$ such that $t$ appears in the
  consequent of $h^{'}$. But $consequent(t) = \{\}$, contradiction.
\end{proof}

\begin{corollary}
  We cannot eliminate an uncommont term $t$ unconditionally once
  $consequent(t) = \{\}$.
\end{corollary}

%%% Local Variables:
%%% mode: latex
%%% TeX-master: "main"
%%% End:


\section{Implementation}

\section{Implementation}

The description of the interpolation algorithm presented in the
previous section suggests a straight forward implementation
of the first two stages. This thesis work considers the following
implementation for the rest of the stages of the algorithm.

\subsection{New optimized conditional elimination step in Kapur's algorithm}

The modification of Phase III implemented in this thesis work
combines and extends the algorithms and data structures from 
the H TODO: continue working here

First, we explain a high level ideal on how 
we improve the \emph{conditional elimination}
step in Kapur's algorithm. We notice that this 
step \emph{propagates equationally} the
head equations of grounded Horn clauses with 
common antecedents. Initially we employ the
unsatisfiability algorithm for Horn clauses 
to achieve such propagation. However, the original
algorithm will not be enough because it will 
only propagate the head equation when all the
antecedents have truth value equal to 
true. To fix that problem, we modify two steps in Gallier's
algorithms:

\begin{itemize}
  \item When we build the data structure \emph{numargs} that keeps 
    track of the number of unproven
    equations in the antecedent of each Horn clause, we change 
    this number by the number
    of unproven uncommon equations in the antecedent of each 
    Horn clause. This will be useful
    because we only introduce head equations intro the queue 
    data structure in Gallier's algorithm
    when all the antecedents are true. With this 
    modification, our algorithm introduces head equations
    when all the antecedent equations are common. 
    Additionally the algorithm can still update
    correctly the truth value of common equations, 
    but these are not relevant for our propagation
    purposes.
  \item To guarantee that \emph{numargs} keeps the 
    right number of uncommon equations yet to
    be proven, we also modify the update mechanism for 
    \emph{numargs} in the main while loop of the algorithm.
    The original algorithm reduces by one the 
    corresponding entry in \emph{numargs}
    whenever a recently popped element from the queue 
    matches the antecedent of a Horn clause. We only
    decrease this value if such popped equation is uncommon. 
    This prevents the algorithm from accidentally
    reducing the number of uncommon equations yet to be proven, 
    which can cause that we propagate the
    uncommon head equation when the antecedent of a Horn 
    clause only consists of common equations.
\end{itemize}

At the end of this algorithm we can identify 
\emph{usable Horn clauses} by checking the Horn clauses
with \emph{numargs} entries equal to 0. Nonetheless, these 
Horn clauses are not the
desired \emph{usable Horn clauses} because the 
unsatisfiability testing algorithm
did not update the antecedents of the Horn clauses. 
The main difficulty to design a data structure
for the latter to work inside the unsatisfiability 
testing algorithm was the queue data structure
only adds grounded equation whenever the truth 
value of the literal changes to true, which happens
during \emph{equational propagation} or during 
the \emph{implicational propagation} steps.
For the \emph{implicational propagation} the task is 
easy because we can know the clause
where the just new proven ground equation comes, 
but it cannot be the same situation
for the \emph{equational propagation} since this 
step relies on congruence closure.

To remedy this issue, we equip our congruence closure 
algorithm with the Explanation operator, so
we can recover the grounded equations needed to entail 
any particular grounded equation. Additionally,
this will require a data structure to maintain the Horn 
clauses for each grounded equation that
it is the head equation of. With the latter we can 
recover the Horn Clauses where each grounded
equation came from to update the antecedents and 
obtain \emph{usable Horn clauses}.

The algorithm appears below in pseudo-code notation:

\begin{algorithm}[!ht]
  \caption{Modified Unsatisfiability Testing for Ground Horn Clauses}
  \linespread{\separationline}\selectfont
  \begin{algorithmic}[1]
    \Procedure {satisfiable}{var H : Hornclause; var queue, combine: queuetype; 
    var GT(H) : Graph; var consistent : boolean}
    \While {queue not empty and consistent}
    \State node := pop(queue);
    \For {clause1 in H[node].clauselist}
    \If {$\neg$ clause1.isCommon()}
    \State {numargs[clause1] := numargs[clause1] - 1}
    \EndIf
    \If {numargs[clause1] = 0}
    \State nextnode := poslitlist[clause1];
    \If {$\neg$ H[nextnode].val}
    \If {nextnode $\neq \bot$ }
    \State {queue := push(nextnode, queue);}
    \State {H[nextnode].val := true;}
    \State {u := left(H[nextnode].atom);} 
    \State {v := right(H[nextnode].atom);}
    \If {FIND(R, u) $\neq$ FIND(R, v)}
    \State {combine := push((u, v), combine);}
    \EndIf
    \Else
    \State {consistent := false;}
    \EndIf
    \EndIf
    \EndIf
    \EndFor
    \If {queue is empty and consistent}
    \State {closure(combine, queue, R);}
    \EndIf
    \EndWhile
    \EndProcedure
    \Statex
    \Procedure {closure}{var combine, queue : queuetype; 
    var R : partition}
    \While {combine is not empty}
    \State (u, v) = pop(combine)
    \State MERGE(R, u, v, queue)
    \EndWhile
    \EndProcedure
  \end{algorithmic}
\end{algorithm}

\begin{algorithm}[!ht]
  \caption{Modified Congruence Closure with Explanation Algorithms - Merge}
  \linespread{\separationline}\selectfont
  \begin{algorithmic}[2]
    \Procedure {Merge}{R : partition, u, v : node; queue, combine : queuetype}
    \If {u and v are constants a and b}
    \State {add a = b to Pending;} 
    \State {Propagate();}
    \Else \Comment {u=v is of the form apply(a1, a2)=a}
    \If {Lookup(Representative(a1), Representative(a2)) is some apply(b1, b2)=b}
    \State {add (apply(a1, a2)=a, apply(b1, b2) = b) to Pending;} 
    \State {Propagate();}
    \Else
    \State {set Lookup(Representative(a1), Representative(a2)) to apply(a1, a2)=a;}
    \State {add apply(a1, a2)=a to UseList(Representative(a1)) and to UseList(Representative(a2));}
    \EndIf
    \EndIf
    \EndProcedure
  \end{algorithmic}
\end{algorithm}

\begin{algorithm}[!ht]
  \caption{Modified Congruence Closure with Explanation Algorithms - Propagate}
  \linespread{\separationline}\selectfont
  \begin{algorithmic}[2]
    \Procedure {Propagate} {\text{ }}
    \While {Pending is non-empty}
    \State {Remove E of the form a=b or (apply(a1, a2) = a, apply(b1, b2) = b) from Pending}
    \If {$Representative(a) \neq Representative(b)$ and w.l.o.g. $|ClassList(Representative(a))| \leq |ClassList(Representative(b))|$}
    \State {oldReprA := Representative(a);}
    \State {Insert edge $a \rightarrow b$ labelled with E into the proof forest;}
    \For {each c in ClassList(oldReprA)}
    \State {set Representative(c) to Representative(b)}
    \State {move c from ClassList(oldReprA) to ClassList(Representative(b))}
    \For {each pointer L in ClassList(u)}
    \If {H[L].val = false}
    \State {set the field H[L].lclass or H[L].rclass pointed to by p to Representative(b)}
    \If {H[L].lclass = H[L].rclass}
    \State {queue := push(L, queue);}
    \State {H[L].val := true}
    \EndIf
    \EndIf
    \EndFor
    \EndFor
    \For {each apply(c1, c2) = c in UseList(oldReprA)}
    \If {Lookup(Representative(c1), Representative(c2)) is some apply(d1, d2) = d}
    \State {add (apply(c1, c2) = c, apply(d1, d2) = d) to Pending;}
    \State {remove apply(c1, c2) = c from UseList(oldReprA);}
    \Else
    \State {set Lookup(Representative(c1), Representative(c2)) to apply(c1, c2) = c;}
    \State {move apply(c1, c2) = c from UseList(oldReprA) to UseList(Representative(b));}
    \EndIf
    \EndFor 
    \EndIf
    \EndWhile
    \EndProcedure
  \end{algorithmic}
\end{algorithm}

\subsection{Ground Horn Clauses with Explanations}

We notice that, by removing our changes to the unsatisfiability testing
for grounded Horn clauses regarding uncommon symbols, we effectively combine
the congruence closure with explanations to the original unsatisfiability
testing algorithm. With the latter, we can query the membership of a Horn
clauses in a given user-defined theory and additionally obtain a proof of
the latter. This approach works by introducing the antecedent equations of
a grounded Horn clause as part of the user-defined theory in order to prove
its head equation. By the Deduction Theorem \cite{10.5555/1642730}, we can
recover a proof of the original queried Horn clause by removing the antecedent
equations appearing the proof given by the Explain operation.

\subsection{Conditional propagation in Kapur's algorithm}

Once the conditional congruence closure data structure is built after 
the execution of the previous step, we can compute conditional eliminations
as follows. For the latter, we will require the following auxiliary functions:

\begin{algorithm}[!ht]
  \caption{Auxiliary function - Candidates}
  \linespread{\separationline}\selectfont
  \begin{algorithmic}[2]
    \Procedure {Candidates} {z3::expr const \& t}
    \If {t is common}
    \State {\textbf{return} $\{ t \}$}
    \Else
    \State {\textbf{return} $\{ t^{'} | t^{'}  \in Class(t), t^{'} \text{is common} \}$}
    \EndIf
    \EndProcedure
  \end{algorithmic}
\end{algorithm}

\begin{algorithm}[!ht]
  \caption{Auxiliary function - Auxiliar explain}
  \linespread{\separationline}\selectfont
  \begin{algorithmic}[2]
    \Procedure {Explain} {z3::expr const \& t1, z3::expr const \& t2}
    \State {z3::expr\_vector ans;}
    \If {t1.id() equals t2.id()}
    \State {\textbf{return} ans;}
    \EndIf
    \State {auto partial\_explain = hsat.equiv\_class.explain(t1, t2);} 
    \For {(auto const \& element : partial\_explain)}
    \If {element is common}
    \State {ans.push\_back(element);}
    \Else
    \State {auto const \& entry = hsat.head\_term\_indexer.find(equation.id());}
    \If {entry equals hsat.head\_term\_indexer.end()}
    \If {equation is common}
    \State {ans.push\_back(equation)};
    \EndIf
    \Else
    \For {(auto const \& hsat\_equation : entry $\rightarrow$ second $\rightarrow$ getAntecedent())}
    \State {ans.push\_back(hsat\_equation);}
    \EndFor
    \EndIf
    \EndIf
    \EndFor
    \EndProcedure
  \end{algorithmic}
\end{algorithm}

\begin{algorithm}[!ht]
  \caption{Auxiliary function - allCandidates}
  \linespread{\separationline}\selectfont
  \begin{algorithmic}[2]
    \Procedure {allCandidates} {z3::expr const \& t}
    \If {t has f-symbol uncommon}
    \State {\textbf{return} $\{\{\}\}$;}
    \EndIf
    \If {t has f-symbol common and is of the form $f(t_1, \dots, t_n)$}
    \State {\textbf{return} $\{candidates(t_1), \dots, candidates(t_n)\}$;}
    \EndIf
    \If {t is a constant}
    \State {undefined}
    \EndIf
    \EndProcedure
  \end{algorithmic}
\end{algorithm}

\begin{algorithm}[!ht]
  \caption{Conditional Elimination - Part 1}
  \linespread{\separationline}\selectfont
  \begin{algorithmic}[2]
    \Procedure {Conditional Elimination} {z3::expr const \& x, z3::expr const \& y}

    \If {x is constant and y is constant}
    \For {$\sigma_x$ in CANDIDATES(x)}
    \For {$\sigma_y$ in CANDIDATES(y)}
    \State {horn\_clause.add(EXPLAIN(x, $\sigma_x$) + EXPLAIN(y, $\sigma_y$), $\sigma_x = \sigma_y$)}
    \EndFor
    \EndFor
    \EndIf

    \If {x is constant and y is of the form $f_y(t_1^{'}, \dots, t_{k_2}^{'})$}
    \For {$\sigma_x$ in CANDIDATES(x)}
    \For {$\sigma_{f_y}$ in CANDIDATES($f_y(t_1^{'}, \dots, t_{k_2}^{'})$)}
    \State {horn\_clause.add(EXPLAIN(x, $\sigma_x$) + EXPLAIN($f_y(t_1^{'}, \dots, t_{k_2}^{'})$), $\sigma_y$), $\sigma_x = \sigma_{f_y}$}
    \EndFor
    \For {$arguments_{f_y}$ in CARTESIANPROD(ALLCANDIDATES($f_y(t_1^{'}, \dots, t_{k_2}^{'})$))}
    \State {horn\_clause.add(EXPLAIN(x, $\sigma_x$) + $\sum_{i=1}^{k_2}$ EXPLAIN($t_i^{'}$, $arguments_{f_y}[i]$), $\sigma_x = f_y(arguments_{f_y})$)}
    \EndFor
    \EndFor
    \EndIf

    \If {x is of the form $f_x(t_1, \dots, t_{k_1})$ and y is a constant}
    \State {\textbf{return} CONDITIONAL ELIMINATION(y, x);}
    \EndIf

    \EndProcedure
  \end{algorithmic}
\end{algorithm}

\begin{algorithm}[!ht]
  \caption{Conditional Elimination - Part 2}
  \linespread{\separationline}\selectfont
  \begin{algorithmic}[2]
    \Procedure {Conditional Elimination} {z3::expr const \& x, z3::expr const \& y}

    \If {x is of the form $f_x(t_1, \dots, t_{k_1})$ and y is of the form $f_y(t_1^{'}, \dots, t_{k_2}^{'})$}
    \For {$\sigma_{f_x}$ in CANDIDATES($f_x(t_1, \dots, t_{k_1})$)}
    \For {$\sigma_{f_y}$ in CANDIDATES($f_y(t_1^{'}, \dots, t_{k_2}^{'})$)}
    \State {horn\_clause.add(EXPLAIN($f_x(t_1, \dots, t_{k_1})$, $\sigma_{f_x}$) + EXPLAIN($f_y(t_1^{'}, \dots, t_{k_2}^{'})$), $\sigma_y$), $\sigma_{f_x} = \sigma_{f_y}$}
    \EndFor
    \For {$arguments_{f_y}$ in CARTESIANPROD(ALLCANDIDATES($f_y(t_1^{'}, \dots, t_{k_2}^{'})$))}
    \State {horn\_clause.add(EXPLAIN($f_x(t_1, \dots, t_{k_1})$, $\sigma_{f_x}$) + $\sum_{i=1}^{k_2}$ EXPLAIN($t_i^{'}$, $arguments_{f_y}[i]$), $\sigma_{f_x} = f_y(arguments_{f_y})$)}
    \EndFor
    \EndFor

    \For {$arguments_{f_x}$ in CARTESIANPROD(ALLCANDIDATES($f_x(t_1, \dots, t_{k_1})$))}
    \For {$\sigma_{f_y}$ in CANDIDATES($f_y(t_1^{'}, \dots, t_{k_2}^{'})$)}
    \State {horn\_clause.add($\sum_{i=1}^{k_1}$ EXPLAIN($t_i$, $arguments_{f_x}[i]$) + EXPLAIN($f_y(t_1^{'}, \dots, t_{k_2}^{'})$), $\sigma_y$), $f_x(arguments_{f_x}) = \sigma_{f_y}$}
    \EndFor
    \For {$arguments_{f_y}$ in CARTESIANPROD(ALLCANDIDATES($f_y(t_1^{'}, \dots, t_{k_2}^{'})$))}
    \State {horn\_clause.add($\sum_{i=1}^{k_1}$ EXPLAIN($t_i$, $arguments_{f_x}[i]$) + $\sum_{i=1}^{k_2}$ EXPLAIN($t_i^{'}$, $arguments_{f_y}[i]$), $f_x(arguments_{f_x}) = f_y(arguments_{f_y})$)}
    \EndFor
    \EndFor

    \EndIf

    \EndProcedure
  \end{algorithmic}
\end{algorithm}

The conditional propagation algorithm produces common Horn clauses from previous
uncommon equations and uncommon Horn clauses obtained in previous steps of Kapur's
algorithm. The are two main invariants for the Horn clauses formed by the
conditional propagation procedure: 

\begin{itemize}
  \item the antecedents are constructed using the \emph{Explain} operator, 
    which returns a sequence of common equations since these are the 
    only merged terms added at the beginning of the initialization 
    routine of the modified Gallier's data structure.

    An additional property must be checked when processing previous
    Horn clauses. The `previous' antecedent for these clauses must be \emph{explainable},
    which means that every equation in the antecedent must belong to the conditional
    congruence closure, otherwise an empty explaination will be produced by the
    \emph{Explain} operator. If a Horn clause antecedent is not \emph{explainable}
    the resulting Horn clause cannot be added to the final result.
    
  \item the consequents are constructed with the help of the \emph{Candidates},
    \emph{AllCandidates}, and the \emph{CartesianProd} operators. 
    The former is a function that takes as input
    a term and returns a list of common terms
    that belong to the equivalence relation of a given term if such term is uncommon,
    and return a list containing the term itself if the term is common; the 
    \emph{AllCandidates} function takes as input a function application term and
    constructs a list of lists with common terms that are equivalent to the arguments
    of the function application if the function symbol is common, and returns an empty
    list otherwise; \emph{CartesianProd} implements a cartesian product, i.e. it 
    produces a list of n-tuples given as input a list of $n$ list of terms (the 
      composition of \emph{CartesianProd} and \emph{AllCandidates} are meant to be
    used to produce a function application free of uncommon terms). Finally, the
    head equations for the new Horn clauses are obtained by equating the left-hand
    side common candidates and right-hand side common candidates obtained by these 
    procedures.

\end{itemize}

%%% Local Variables:
%%% mode: latex
%%% TeX-master: "main"
%%% End:


\section{Evaluation}

TODO: keep working here.

%%% Local Variables:
%%% mode: latex
%%% TeX-master: "main"
%%% End:
