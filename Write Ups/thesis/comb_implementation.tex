\section{Implementation}

The implementation mantains a map data structure that 
keeps track of the \emph{partial interpolants}. This ensures
that the base case for the above formula $p(c)$ is replaced
by previous clauses as required in \cite{10.1007/11532231_26}.

Since introducting negations is necessary to compute partial interpolants,
we noticed the following interaction with the theories involved in the
thesis work:

\begin{itemize}
  \item EUF case: negations of literals in this theory are just
    disequalities, which the interpolantion algorithm implemented
    handles as Horn clauses with a false head term.
  \item UTVPI case: negations of literals in this theory are either
    disequalities or strict inequalities. The disequalities are purified
    an appended to the EUF component, and the strict inequalities are 
    appended as a disjunction of two non-strict inequalities following
    the axioms of number theory with addition and a total order relation
    \cite{DBLP:books/daglib/0076838}.
\end{itemize}

The main loop of the procedure is shown below:

\begin{algorithm}[!ht] \label{thcomb_propagation_loop}
  \caption{ Nelson-Oppen Propagation }
  \linespread{\separationline}\selectfont
  \begin{algorithmic}[2]
    \Procedure { Nelson-Oppen Propagation  } { 
      z3::expr\_vector const \& part\_A, 
      z3::expr\_vector const \& part\_B 
  }

    \State { $T_1, T_2 = Purify(part\_A, part\_B)$}

    \State DisjunctionEqualitiesIterator $\psi()$
    \State $\psi.init()$
    \While {true}

    \If { $T_1 \models_{EUF} \bot $ }
    \State { return $T_1$ }
    \EndIf
    \If { $T_2 \models_{UTVPI} \bot $ }
    \State { return $T_2$ }
    \EndIf

    \If { $T_1 \models_{EUF} \psi.current()$ }

    \If { $T_2 \models_{UTVPI} \psi.current()$ }
    \State { continue }
    \Else
    \State { append $\psi.current()$ to $T_2$ }
    \State { $\psi.init()$ }
    \EndIf

    \Else

    \If { $T_2 \models_{UTVPI} \psi.current()$ }
    \State { continue }
    \Else
    \State { append $\psi.current()$ to $T_2$ }
    \State { $\psi.init()$ }
    \EndIf

    \EndIf

    \State $UpdatePartialInterpolant(\psi.current())$
    \State $\psi.next()$

    \EndWhile

    \EndProcedure
  \end{algorithmic}
\end{algorithm}

%%% Local Variables:
%%% mode: latex
%%% TeX-master: "main"
%%% End:
