% Example template for using the unmeethesis style
% This example is for a Master's candidate in Mathematics
% It contains examples of front matter and most sections that the
% typical graduate student would need to include
% By: N. Doren 02/10/00
%     Minor mods by N. Doren 08/26/11

% Use the following specification for BOTTOM page numbering:
\documentclass[botnum, fleqn]{unmeethesis}
                 % OR
% Use the following specification for TOP page numbering:
% \documentclass[fleqn]{unmeethesis}

\begin{document}

\frontmatter

% Uncomment the next command if you see weird paragraph spacing:
% That is, if you see paragraphs float with lots of white space
% in between them:

% \setlength{\parskip}{0.30cm}


\title{An Awesome Thesis That Will Prove \\ to the Universe
       That I Really Deserve This Honorable Degree}

\author{Albert Richard Einstein, III}

\degreesubject{M.S., Mathematics}

\degree{Master of Science \\ Mathematics}

\documenttype{Thesis}

\previousdegrees{A.A.S., University of Southern Swampland, 1988 \\
                 M.S., Art Therapy, University of New Mexico, 1991}

\date{December, \thisyear}

\maketitle

%\makecopyright
%Copyright page is no longer necessary D. Murrell

\begin{dedication}
   To my parents, Albert II and Gladys, for their support,
   encouragement and the Corvette they're giving me for graduation. \\[3ex]
   ``A bird in hand is worth two in the bush''
         -- Anonymous
\end{dedication}

\begin{acknowledgments}
   \vspace{1.1in}
   I would like to thank my advisor, Professor Martin Sheen, for his support
   and some great action movies.  I would also like to thank my dog, Spot,
   who only ate my homework two or three times.  I have several other people
   I would like to thank, as well.\footnote{To my brother and sister, who
   are really cool.}
\end{acknowledgments}

\maketitleabstract %(required even though there's no abstract title anymore)

\begin{abstract}
   The theory of relativity is a real ``toughie'' to prove, but with the
   help of my family and my great grandpa Al, this paper presents the
   proof in its entirety.  Most of the math is correct, and the
   part about ``warp speed'' and ``parallel universe'' sounds very high-tech.
\clearpage %(required for 1-page abstract)
\end{abstract}

\tableofcontents
\listoffigures
\listoftables

\begin{glossary}{Longest  string}
   \item[$a_{lm}$]
      Taylor series coefficients, where $l,m = \{0..2\}$
   \item[$A_{\bf{p}}$]
      Complex-valued scalar denoting the amplitude and phase.
   \item[$A^T$]
      Transpose of some relativity matrix.
\end{glossary}

\mainmatter

\chapter{Introduction}
\section{\label{section:overview}Overview}
   The classic approach to proving a theorem is some really difficult 
   mathematics.  For the theory of relativity, I asked grandpa Al exactly 
   how he proved it.  He gave me a few hints, including some stuff about
   rest mass and big electro-motive force.  I think he is really smart.
\section{Conclusions}
   I conclude that this is a really short thesis.

\chapter{Future Work}
   I'm sure my future work will consist of lots of other famous stuff.

\chapter*{Appendices}

\addcontentsline{toc}{chapter}{Appendices}
 % Next lines duplicated from .toc file and used to create mini
 % "Appendix Table of Contents," if desired:
   \contentsline {chapter}{\numberline {A}Proving $E=MC^2$}{4}
   \contentsline {chapter}{\numberline {B}Derivation of $A = \pi r^2$}{5}
 % End mini table of contents

\appendix
\chapter{Proving $E=MC^2$}
   I refer the reader to many of grandpa's famous books on this subject.
\chapter{Derivation of $A = \pi r^2$}
   A circle is really a square without corners.  QED.

%\bibliographystyle{AMS}
%\bibliography{bibfile_name}

\end{document}
