\section{Conclusions}

This chapter discussed the new approach
for interpolant generation introduced by
Prof. Kapur for the EUF theory. We 
proposed a modification of his algorithm
and proved its correctness. The 
implementation and testing work confirms 
that the approach produces stronger interpolants
compared to other interpolant generation
algorithms (iZ3, Mathsat).

The performance comparison section indicates
that of our implementation appears to be slower 
than iZ3 and Mathsat.
This can be due the quadratic step 
by the Phase II in Kapur's algorithm
and both iZ3 and Mathsat might be able
to find the linear refutational proof 
discussed in \ref{performance_test_lemma}.

On the other hand, it is important to 
highlight that the algorithm have different
output specification since Prof. Kapur algorithm
is able to find a uniform interpolant.

There are several places
where the current implementation can be
improved. Currently, the data structures
for flattening introduce additional constants
that might not be needed since the implementation
used a recursive scan of the arguments of each
expression without checking for term sharing.
Moreover, the explanation mechanism might be
improved/changed by the proof producing mechanism
already available in Z3. 

%%% Local Variables:
%%% mode: latex
%%% TeX-master: "main"
%%% End:
