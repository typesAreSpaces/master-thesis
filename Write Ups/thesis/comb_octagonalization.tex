\section{Octagonalization of input formula}

This section discusses a minor issue that occurs
after purifiying an input formula from the combined
language of any theory with UTVPI. It is trivial
to see that, in the language of UTVPI, any linear 
combination of more than 2 variables does not belong
to the UTVPI language by definition. On the other hand,
purification will introduce equations with an additional 
fresh variable, so some valid input formulas in the 
UTVPI language might not be recognized as such.

\begin{example} \label{example_octagonalization_issue}
  Let us consider the following A-part 
  input formula for the
  combined theory of EUF and UTVPI: 
  $f(x) + g(y, 29) \leq 120$.

  The formula introcuded by
  purification does not conform the UTVPI language:
  $\mathfrak{a_1} = g(y, 29), 
  \mathfrak{a_2} = f(x), 
  \mathfrak{a_3} = \mathfrak{a_1} + \mathfrak{a_2},
  \mathfrak{a_3} \leq 120$
\end{example}

In some cases, the purified formula can be modified by 
propagating/eliminating unnecessary constants introduced
by purification. An equisatisfiable formula to the 
above shown in example \ref{example_octagonalization_issue}:
$\mathfrak{a_1} = g(y, 29), 
\mathfrak{a_2} = f(x), 
\mathfrak{a_1} + \mathfrak{a_2} \leq 120$.

\begin{definition}
 An input formula in the combined language of theories including
 the $UTVPI$ theory is \emph{octagonizable} if the $UTVPI$
 component of the purified formula conforms the restrictions
 of the $UTVPI$ language.
\end{definition}

\begin{lemma}
  An input formula $\varphi$ in the combined language 
  of $EUF$ and $UTVPI$ is \emph{octanizable} if 
  and only if for every subterm of $\varphi$, the purified
  equation introduced by purification containing three 
  constants can be replaced by an equivalent equation
  with only two constants.
\end{lemma}

\begin{proof}
  The left to right implication is straightforward to prove 
  since there are no equalities with three constants and 
  the equalities with at most two constants can be replaced
  by two inequalities with at most two constants, thus the formula
  comforms the language constraint in the UTVPI theory.

  Let us prove the right to left implication. 
  Since the EUF language does not contain addition nor subtraction,
  all the equalities produced by purification will contain at most
  three constanst. By hypothesis, these equations can be replaced
  by equations with at most two constants, which can be replaced
  by two inequalities respectively by the UTVPI theory.
  Hence the UTVPI part of the purified output
  in additions with the latter inequalities conform the UTVPI
  language constraint.
\end{proof}

Our implementation propagates constants in the integer domain 
and eliminates unnecessary constants using the Z3 tactic
called \emph{solver-eqs}. The tactic solver-eqs eliminates 
variables using Gaussian elimination \cite{z3-strategies}.
This transformation happens before the formula is applied to
the UTVPI interpolant algorithm.

%%% Local Variables:
%%% mode: latex
%%% TeX-master: "main"
%%% End:
