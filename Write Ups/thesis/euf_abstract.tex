Interpolation algorithms for the theory of equality 
with uninterpreted functions are relevant as the core 
component of verification algorithms. Many useful techniques 
in software engineering like bounded/unbounded model 
checking and invariant generation benefit directly from 
this technique. In \cite{10.1007/3-540-58179-0_44}, the 
authors introduced a methodology to debug/verify the 
control logic of pipelined microprocessors by encoding 
its specification and a logical formula denoting the 
implementation of the circuit into a EUF solver.

Previous work addressing the interpolation problem for 
EUF has involved techniques ranging from
interpolant-extraction from refutation proof 
trees \cite{10.1007/978-3-540-24730-2_2, mcmillan2011interpolants, 
10.1007/978-3-642-31612-8_24}, and colored congruence closure
graphs \cite{10.1007/978-3-642-00768-2_34}. Kapur's algorithm 
uses a different approach by using approximated
quantifier-elimination, a procedure that given a formula, 
it produces a logically equivalent formula
 without a variable in particular \cite{DBLP:books/daglib/0076838}.

%%% Local Variables:
%%% mode: latex
%%% TeX-master: "main"
%%% End:
