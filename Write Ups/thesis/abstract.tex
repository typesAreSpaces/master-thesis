\begin{abstract}
  This thesis discusses the implementation
  for the uniform interpolation problem of the 
  following theories: (quantifier-free) equalitiy with uninterpreted
  functions (EUF), unit two variable per 
  inequality (UTVPI), and theoretical aspects for the 
  combination of the two previous theories. 
  The uniform interpolation algorithms implemented in this thesis were 
  originally proposed in \cite{KAPUR2017}. 

  Refutational proof-based solutions are the usual approach 
  of many interpolation algorithms 
  \cite{10.1007/978-3-642-00768-2_34, mcmillan2011interpolants, 10.1007/978-3-540-24730-2_2}. 
  The approach taken in \cite{KAPUR2017} relies on quantifier-elimination heuristics 
  to construct a uniform interpolant using one of the two formulas involved 
  in the interpolation problem. The latter makes possible to study the complexity 
  of the algorithms obtained. 
  On the other hand, the combination method implemented 
  in this thesis uses a Nelson-Oppen framework, thus we still require for 
  this particular situation a refutational proof in order to guide the 
  construction of the interpolant for the combined theory.

  The implementation uses Z3 \cite{10.1007/978-3-540-78800-3_24} for parsing 
  purposes and satisfiability checking in the combination component of the 
  thesis. Minor modifications were  applied to the Z3's enode data structure 
  in order to label and distinguish formulas efficiently (i.e. distinguish 
  A-part, B-part). Thus, the project can easily be integrated to the Z3 solver 
  to extend its functionality for verification purposes using the Z3 plug-in module.

\clearpage %(required for 1-page abstract)
\end{abstract}


%%% Local Variables:
%%% mode: latex
%%% TeX-master: "main"
%%% End:
