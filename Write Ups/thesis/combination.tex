\chapter{Interpolation algorithm for the theory combination of EUF and UTVPI}

Theory combination is an approach in order to reuse the veriffication algorithms
for the theories used in a problem.  

\section{Algorithm}

TODO. Discuss relevance of SAT solver and resolution proof. 
TODO. Mention issues with the negation for literals in UTVPI due to partial interpolantion.

\section{Implementation}

TODO.

\begin{algorithm}
  \caption{ Nelson-Oppen Propagation }
  \linespread{\separationline}\selectfont
  \begin{algorithmic}[2]
    \Procedure { Nelson-Oppen Propagation  } { z3::expr\_vector const \& part\_A, z3::expr\_vector const \& part\_B }

    \State { $T_1, T_2 = Purify(part\_A, part\_B)$}

    \State DisjunctionEqualitiesIterator $\psi()$
    \State $\psi.init()$
    \While {true}

    \If { $T_1 \models_{EUF} \bot $ }
    \State { return $T_1$ }
    \EndIf
    \If { $T_2 \models_{UTVPI} \bot $ }
    \State { return $T_2$ }
    \EndIf

    \If { $T_1 \models_{EUF} \psi.current()$ }

    \If { $T_2 \models_{UTVPI} \psi.current()$ }
    \State { continue }
    \Else
    \State { append $\psi.current()$ to $T_2$ }
    \State { $\psi.init()$ }
    \EndIf

    \Else

    \If { $T_2 \models_{UTVPI} \psi.current()$ }
    \State { continue }
    \Else
    \State { append $\psi.current()$ to $T_2$ }
    \State { $\psi.init()$ }
    \EndIf

    \EndIf

    \State $\psi.next()$

    \EndWhile

    \EndProcedure
  \end{algorithmic}
\end{algorithm}

\section{Evaluation}
TODO.

%%% Local Variables:
%%% mode: latex
%%% TeX-master: "main"
%%% End:
