\section{Yorsh-Musuvathi Interpolation Combination Framework}

The algorithm extends a Nelson-Oppen framework which allows the Yorsh-Musuvathi
framework to produce an Interpolant
for combined theories using Interpolant
generation algorithm for each of the 
theories involved. 
In order to 
integrate interpolants to the latter, 
the algorithm includes a 
\emph{partial interpolant} every time a disjunction of equalities 
(conflict clause) is propagated. 
These \emph{partial interpolants} are 
computed from an unsatisfiability proof obtained by including the 
negation of the disjunction to the formula using the following
definition: 

\begin{definition} \cite{10.1007/11532231_26}
  Let $\langle A, B \rangle$ be a pair of clause sets such
  that $A \land B \vdash \bot$. Let $\cal{T}$ be a proof of
  unsatisfiability of $A \land B$. The propositional 
  formula $p(c)$ for a clause $c$ in $\cal{T}$ is defined
  by induction on the proof structure:
  \begin{itemize}
    \item if $c$ is one of the input clauses then
      \begin{itemize}
        \item if $c \in A$, then $p(c) := \bot$
        \item if $c \in B$, then $p(c) := \top$
      \end{itemize}
    \item otherwise, $c$ is a result of resolution, i.e. 
      let $c_1, c_2$ be two clauses of the form $x \lor c_1^{'}$,
      $\neg x \lor c_2^{'}$ respectively. The partial interpolant 
      for $c$ is defined as follows:
      \begin{itemize}
        \item if $x \in A$ and $x \not \in B$ (x is A-local), then $p(c) := p(c_1) \lor p(c_2)$
        \item if $x \not \in A$ and $x \in B$ (x is B-local), then $p(c) := p(c_1) \land p(c_2)$
        \item otherwise (x is AB-common), then 
          $p(c) := (x \lor p(c_1)) \land (\neg x \lor p(c_2))$
      \end{itemize}
  \end{itemize}
\end{definition}

%%% Local Variables:
%%% mode: latex
%%% TeX-master: "main"
%%% End:
