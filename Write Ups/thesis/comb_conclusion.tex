\section{Conclusions}

This chapter showed the implementation of the
interpolation combination algorithm in 
\cite{10.1007/11532231_26} and the necessary
changes for the EUF and UTVPI algorithms
presented in previous chapters.

The performance comparison shows a slower performance
of the implementation with respect to iZ3 and Mathsat
when the common check is not enable. The latter was 
done on purpose to see the behaviour of the implementation
on the interpolantion combination component.
On the other hand, if the common check is enabled then
our implementation is faster than the algorithms in iZ3 and
Mathsat. The latter suggests that iZ3 and Mathsat might
not check for common symbols globally and the computation 
to distinguish symbols is done locally. 

Currently the implementation does not incrementally the 
consistency of the current formula in the main loop of 
the implementation \ref{thcomb_propagation_loop}. 
Changing the latter to an incrementally approach will
maintain lemmas so the computation of intermediate results
will not be performed repetitively. 

Additionally, recent approaches in theory combination
might benefit the interpolation combination problem. 
In \cite{10.1007/978-3-642-22119-4_1} it was proved that
using a model-based combination approach might not require
the propagation of disjunction since this happens internally
at the core level of the model-based approach. This approach
was not investigated in the thesis work since it required
to modify the SMT solver at a non supercifial level.

%%% Local Variables:
%%% mode: latex
%%% TeX-master: "main"
%%% End:
