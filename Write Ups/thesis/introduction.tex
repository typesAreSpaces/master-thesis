\chapter{Introduction}

Modern society is witness of the impact computer software has done
in recent years. The benefits of this massive automation is endless. On the
other hand, when software fails it becomes a catostrophe to the point that
some industries are severely affected and human's life are threaded 
at worst. 

Due to strict and ambitious agendas, many software products
are shipped with unseen bugs which might potential thread people's life
like in critical systems. Several approaches have been used to improve 
software quality. However, many of these approaches offer partial 
coverage or might take an abyssal amount of human effort to provide 
such solutions and thus cannot be consideral practical. These contributions 
are relevant for certain applications and their proper use fit 
such workflows uniquely. 

Formal methods aim to bring a unique combination of automation, rigor, and 
efficiency (whenever effecient algorithms exits for the verification task
at hand). This thesis discusses a particular problem in software verification
known as \emph{the interpolantion problem} for the theories of the quantifier-free
fragment of equality with uninterpreted functions (EUF) and unit two variable per
inequality (UTVPI). These two theories have been studied extensively and 
researcher have found several applications for them. 

\section{Backgroud}

An interpolant of a pair of two logical formula is a logical formula
such that it is implied by the first formula in the pair and logically
inconsistent with the second, having only common symbols of the two formulas
in the original pair. Informally this means that the interpolant `belongs'
to the consequence of the first formula of such pair, and `avoids' the 
consequence of the second formula. Not surprisingly, this intuition is 
behind many software verification routines where the first formula models
the desirable state (termination, correctness) of a computer program, 
and the second formula models the set of undesirable states (non-termination, errors, crashes,
etc) of such software. In Chapter \ref{preliminaries}, an extensive review
of the formal concept is provided.

Interpolants find two main application in software verification:

\begin{itemize}
  \item refinement of abstract models: TODO. explain the latter
  \item invariant generation: TODO. same
\end{itemize}

\section{Related work}

\begin{itemize}
\item (Theoretical) approaches
Proof-based
Color-based
Brutomesso 
Bonachina
\item (Implementation)
iZ3
\end{itemize}

TODO.

\section{Outline of the thesis}

\begin{itemize}
\item Chapter X is about Y ...
\item and so on and so on ...
\end{itemize}


TODO.

%%% Local Variables:
%%% mode: latex
%%% TeX-master: "main"
%%% End:
