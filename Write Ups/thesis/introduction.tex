\chapter{Introduction}

Modern society is witness of the impact computer software 
has done in recent years. The benefits of this massive 
automation is endless. On the other hand, when software 
fails, it becomes a catostrophe ranging from economic loss
to threats for human life. 

Due to strict and ambitious agendas, many software products
are shipped with unseen and unintentional bugs which might potentially put
at risk people's life like in critical systems. Several approaches have been 
used to improve software quality. However, many of these approaches 
offer partial coverage or might take an abyssal amount of human effort to 
provide such solutions and thus cannot be consideral practical. These 
contributions are relevant for certain applications and their proper 
use fit such workflows uniquely. 

Formal methods aim to bring a unique combination of automation, rigor, and 
efficiency (whenever effecient algorithms exits for the verification task
at hand). This thesis discusses a particular problem in software verification
known as \emph{the interpolantion problem} for the theories of the quantifier-free
fragment of equality with uninterpreted functions (EUF) and unit two variable per
inequality (UTVPI). These two theories have been studied extensively and 
researcher have found several applications for them. 

\section{Backgroud}

An interpolant of a pair of two logical formula is a logical formula
such that it is implied by the first formula in the pair and logically
inconsistent with the second, having only common symbols of the 
two formulas in the original pair. Informally this means that the 
interpolant `belongs' to the consequence of the first formula of 
such pair, and `avoids' the consequence of the second formula. 
Not surprisingly, this intuition is behind many software verification 
routines where the first formula models the desirable state 
(termination, correctness) of a computer program, and the second 
formula models the set of undesirable states (non-termination, 
errors, crashes, etc) of such software. In Chapter 
\ref{preliminaries}, an extensive review of the formal concept 
is provided.

Eventhough interpolants are not a direct concern in verification
problems, these problems are found in the core algorithms of the 
following two applications:

\begin{itemize}
  \item Refinement of abstract models: In order to improve 
    coverage and decrease the complexity in verification problems,
    abstract interpretation has become a proper technique to 
    accomplish the latter together with model checking. Eventhough the
    methodology provides sound results, it is certainly not complete.
    Additionally, several abstractions do not capture the semantical
    meaning of programs due to the \emph{over-approximation} approach.
    Hence, interpolants are used to strength predicate
    abstractions by using interpolants constructed from valid 
    traces in the abstract model but not valid in the actual model
    (\emph{spurious counterexamples}) \cite{10.1145/876638.876643,
    10.1007/978-3-540-45069-6_1, 10.1145/982962.964021}.
  \item Invariant generation: following the same idea as in the previous
    case, if a fix point is obtained in the refinement process 
    we can produce logical invariant of computer programs \cite
    {10.1007/978-3-540-45069-6_1, 4401986}. \footnote{Interpolation
      generation has found applications in a \emph{lazy framework}
      similarly to the SAT/SMT algorithms. Whereas the former is
    about the production of interpolants, and the latter is for 
  assignments/models respectively.}
\end{itemize}

\section{Related work}

There are several algorithms for interpolation for the theories
involved in the thesis work. The approaches can be classified
into the following categories:

\begin{itemize}
  \item Proof-based approach: This category relies on the availability
    of a refutational proof. Linear transformations to the proof
    tree are performed in order to obtain the interpolant.
    KEEP: working here.

  \item Reduction-based approach: TODO.

\end{itemize}

\section{Outline of the thesis}

\begin{itemize}
\item Chapter 2 provides an extensive background 
  of fundamentals ideas, algorithms and decision procedures 
  used in the thesis work.
\item Chapters 3, 4, and 5 explain the implentation aspect
  of the theories EUF, UTVPI, and their combination. These
  chapters share the same structure. They start with the
  algorithms used to solve the interpolation problem, provides
  discussion about implementation details including diagrams
  of the arquitecture of the implemented system, and at the end
  show a performance comparision with the iZ3 interpolation 
  tool available in the SMT solver Z3 until version 4.7.0.
\end{itemize} 

%%% Local Variables:
%%% mode: latex
%%% TeX-master: "main"
%%% End:
