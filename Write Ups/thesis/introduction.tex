\chapter{Introduction}

Modern society is witness of the impact computer software 
has done in recent years. The benefits of this massive 
automation is endless. On the other hand, when software 
fails, it becomes a catastrophe ranging from economic loss
to threats for human life. 

Due to strict and ambitious agendas, many software products
are shipped with unseen and unintentional bugs which might potentially put
at risk people's life like critical systems. Several approaches have been 
used to improve software quality. However, many of these approaches 
offer partial coverage or might take an abyssal amount of human effort to 
provide such solutions. Thus, these approaches cannot be considered practical. 
On the other hand, for certain applications these contributions are relevant
and their proper use fit such work flows uniquely.

Formal methods aim to bring a unique combination of automation, rigor, and 
efficiency (whenever efficient algorithms exits for the verification task). 
This thesis discusses a particular problem in software verification
known as \emph{the interpolation problem} for the theories of the quantifier-free
fragment of equality with uninterpreted functions (EUF) and unit two variable per
inequality (UTVPI) and their combination. These two theories have been 
studied extensively and researcher have found several applications for them. 

\section{Background}

An interpolant of a pair of logical formulas $(\alpha, \beta)$ 
is a logical formula $\gamma$
such that $\alpha$ implies $\gamma$, $\beta \land \gamma$ is logically
inconsistent, and $\gamma$ only has common symbols of $\alpha$ and $\beta$.
Informally this means that the 
interpolant `belongs' to the consequence of $\alpha$, 
and `avoids' being part of the consequence of $\beta$. 
Not surprisingly, this intuition is in software verification 
routines where the first formula models a desirable state/property
(termination, correctness) of a computer program, and the second 
formula models the set of undesirable states (non-termination, 
errors, crashes) of such software. In Chapter 
\ref{preliminaries}, an extensive review of the formal concept 
is provided.

Even though interpolants are not a direct concern in verification
problems, these problems are found in the core algorithms of the 
following two applications:

\begin{itemize}
  \item Refinement of abstract models: In order to improve 
    coverage and decrease the complexity in verification problems,
    abstract interpretation has become a proper technique to 
    accomplish the latter together with model checking. Despite the fact that 
    the methodology provides sound results, it is certainly not complete.
    Additionally, several abstractions do not capture the semantical
    meaning of programs due to the \emph{over-approximation} approach.
    Hence, interpolants are used to strength predicate
    abstractions by using interpolants constructed from valid 
    traces in the abstract model but not valid in the actual model
    (\emph{spurious counterexamples}) \cite{10.1145/876638.876643,
    10.1007/978-3-540-45069-6_1, 10.1145/982962.964021}.
  \item Invariant generation: following the same idea as in the previous
    case, \emph{if a fix point is obtained in the refinement process},
    we can obtain a logical invariant of computer programs 
    \footnote{The interpolation
      generation approach discussed can be understood as a \emph{lazy framework}
      similarly to SAT/SMT algorithms. The former is
      about the production of interpolants, the latter is for 
      assignments/models respectively. Both \emph{block/learn} the formulas
    in order to find their results.}
    Situations where this happens might be due to the finiteness of possible 
    states in the program. It is worth mentioning that the 
    invariant problem as stated in \cite{10.1145/363235.363259} 
    is undecidable \cite{10.2307/1990888, 10.1145/371282.371285}.
\end{itemize}

\section{Related work}

There are several interpolation algorithms for the theories
involved in the thesis work. The approaches can be classified
into the following categories:

\begin{itemize}
  \item Proof-based approach: This category relies on the availability
    of a refutation proof.  The interpolant is constructed using a
    recursive function over the structure of the proof
    tree. In \cite{10.1007/978-3-540-24730-2_2,mcmillan2011interpolants}
    the author defines an interpolation calculus. This particular
    approach uses a proof tree produced by the SMT solver Z3 
    and does not need to modify any of Z3's internal mechanisms.
    Among the advantages of this approach is that
    theory combination is given for free since the SMT solver takes care of this
    problem. On the other hand, Z3's
    satellites theories are not sufficiently integrated with its
    proof-producing mechanism. Hence, one can find $\mathcal{T}-lemmas$
    as black-boxes which introduces incompleteness in the
    interpolation calculus. For completeness, these lemmas are 
    solved separately by another interpolation algorithm for the respective
    theory. 
    In \cite{10.1007/11532231_26} the authors provide a Nelson-Oppen framework
    to compute interpolants. For the convex-case, the approach only 
    exchange equalities as required by the Nelson-Oppen framework. For
    the non-convex case, the authors require a resolution-based refutation proof
    to compute interpolants using Pudlak's algorithm. The introduction of the
    class of \emph{equality interpolating theories} is 
    considered among the most relevant contributions of the paper 
    by the verification community. This is property about theories which states
    that if a theory is capable of proving a mixed equality $a = b$ (an equality
    which contains symbols from the two formulas in the interpolating problem)
    then it exists a common term in the language of the theory $t$ (known as
    the interpolating-term) such that 
    the theory can prove $a = t$ and $t = b$. The property facilities 
    formula-splitting for interpolation purposes.
    In \cite{10.1007/978-3-642-00768-2_34} the authors modify a 
    resolution-based refutation proof by introducing common-terms in 
    the proof in order to produce interpolants in what is called
    colorable-proofs, which are proof trees which do not contain AB-mixed
    literals. This is pointed as an improvement to the approach followed 
    in \cite{10.1007/11532231_26} which executes a similar idea but it is done 
    progressively as the proof-tree is built and does not require 
   \emph{equality propagating} theory solvers
    \footnote{The authors in \cite{10.1007/11532231_26} require that 
      the theory solvers keep track of the interpolating-term 
      and propagate this term whenever
    possible}. 
    However, this results is not generalizable
    to non-convex theories due to internal constraints.

  \item Reduction-based approach: This method transforms the interpolation
    problem into a query for some solver related to the theory. 
    An example of this approach can be found in \cite{10.1007/978-3-540-69738-1_25}
    where the authors use a linear-inequality solver to provide an interpolant
    for the theory of linear inequalities over the rational/real numbers 
    ($LIA(\mathbb{Q})/LIA(\mathbb{R})$). 
    Additionally, they integrate the procedure with a 
    \emph{hierarchical reasoning} approach in order to incorporate 
    the signature of theory for (quantifier-free) equalities with uninterpreted 
    functions.
\end{itemize}

\section{Outline of the thesis}

\begin{itemize}
  \item Chapter 2 provides an extensive background 
    of fundamentals ideas, definitions and decision procedures 
    used in the thesis work.
  \item Chapters 3 and 4 explain implementation details
    of the uniform interpolating algorithms for the theories
    EUF, UTVPI. Chapter 5 discusses the implementation of
    a theory combination algorithm using the uniform 
    interpolation algorithm from Chapter 3 and 4. These
    chapters share the same structure. They start with the
    algorithms used to solve the interpolation problem, 
    discuss implementation details including diagrams
    of the architecture of the implemented system, and 
    show a performance comparison with the iZ3 interpolation 
    tool available in the SMT solver Z3 until version 4.7.0.
\end{itemize} 

\section{Contributions}

The contributions of the thesis can be summarized as follows:

\begin{itemize}
  \item[1.] Implementation of the interpolation algorithm for the theory EUF 
    proposed in \cite{KAPUR2017}.
  \item[2.] Formulation and implementation of a new 
    procedure for checking unsatisfiability of grounded equations in 
    Horn clauses using a congruence closure algorithm with explanations
    used in the implementation of item 1.
  \item[3.] Implementation of the interpolation algorithm for the theory UTVPI
    proposed in \cite{KAPUR2017}.
  \item[4.] Implementation of the combination procedure for interpolating
    algorithm proposed in \cite{10.1007/11532231_26} in order to 
    combine the implementations of item 1 and item 3.
\end{itemize}

%%% Local Variables:
%%% mode: latex
%%% TeX-master: "main"
%%% End:
