\subsection{Nelson-Oppen framework for theory and interpolation
combination}

The theory combination problem consists on taking a 
formula from the union of two (or more) disjoint 
languages and tell if such formula is satisfiable
or not in the combined theory, i.e. a theory resulting
after putting together two (or more) axiomatizations.

In \cite{10.1145/357073.357079} the authors defined a procedure
to achieve the above problem. The key idea is to \emph{purify} (
or separate) the subformulas by including additional constant 
symbols equating subterms such that the resulting formula 
can be splitted into components of the appropriate language 
for each theory solvers to work with. The separation naturally
will hide relevant information to the solvers and they 
might not be able to decide satisfiability correctly.
The authors noticed that to solve the above problem it is enough to 
share disjunction of equalities between the combined theories of shared
terms. In addition, they proved that some theories have the 
following property:

\begin{definition}
  Let $\mathcal{T}$ be a theory. We say that $\mathcal{T}$
  is a \emph{convex theory} if a finite conjunction of formulas 
  in $\mathcal{T}$ $\psi = \bigwedge_{i = 1}^m \psi_i$ satisfies
  $\psi \models_{\mathcal{T}} \bigvee_{j = 1}^n 
  x_j = y_j$, then exists $k \in \{1, \dots, n \}$ such that 
  $\psi \models_{\mathcal{T}} x_k = y_k$.
\end{definition}

Hence, it is important to detect
whether the theories involved are convex or not since 
this can improve performance since convex theories do not
need to share disjunctions of equalities as mentioned before
(since all these disjunctions imply a single equality).

\begin{example}
  \begin{itemize}
    \item The conjunctive fragment of equality logic is convex since
      it can always decide the membership of an equation in the 
      equivalence relation. 
    \item The theory of UTVPI over the intergers is not 
      convex. To see the latter consider
      $1 \leq x \land x \leq 2 \models_{UTVPI(\mathbb{Z})} 1 = x \lor 2 = x$.
      However, it is not the case that 
      $1 \leq x \land x \leq 2 \models_{UTVPI(\mathbb{Z})} 1 = x$
      nor 
      $1 \leq x \land x \leq 2 \models_{UTVPI(\mathbb{Z})} 2 = x$.
  \end{itemize}
\end{example}

An interpolation combination framework as proposed in \cite{10.1007/11532231_26}
follow the same idea towards theory combination. Inductively, they define
\emph{partial interpolants} for each shared equality/disjunction of equalities
until some theory reaches the unsatisfiable state, which is expected since 
an interpolant is a pair a mutually contradicting formulas.

This framework was implemented in the thesis work. This framework was chosen
in particular since it allows to work with non-convex theories (in our case
for the theory of UTVPI over $\mathbb{Z}$).

%%% Local Variables:
%%% mode: latex
%%% TeX-master: "main"
%%% End:
