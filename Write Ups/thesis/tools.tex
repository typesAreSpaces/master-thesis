\section{General system description}

The algorithms implemented in this thesis used the C++ programming
language. The overall architecture of the system is the following:

\begin{figure}

  \centering

  \begin{tikzpicture}
    \node[module]                        (I0) {\tiny Input};
    \node[module, right=3mm of I0]       (I1) {\tiny Parser (Z3)};
    \node[module, right=3mm of I1]       (I2) {\tiny Preprocessor};
    \node[module, right=23mm of I2]      (I4) {\tiny Sat engine (PicoSAT/TraceCheck)};
    \node[module, above=2mm of I4]       (I3) {\tiny Theory Combination Interpolator};
    \node[module, below left=3mm of I4]  (I5) {\tiny EUF interpolator};
    \node[module, below right=3mm of I4] (I6) {\tiny UTVPI interpolator};
    \node[module, right=24mm of I4] (I7) {\tiny Output};
    \foreach \i in {5, 6}
    \draw[<->, dashed] (I4)--(I\i);
    \node[fit=(I4) (I5) (I6), draw, inner sep=0.5mm]           (fitThCombInternal) {};
    \node[fit=(I3) (fitThCombInternal), draw, inner sep=0.5mm] (fitThComb)         {};
    \draw[->] (I0)--(I1);
    \draw[->] (I1)--(I2);
    \draw[->] (I2)--(I4);
    \draw[->] (I2)--(I4);
    \draw[->] (I4)--(I7);
  \end{tikzpicture}

  \caption{General System Diagram} \label{general_architecture}

\end{figure}

%%% Local Variables:
%%% mode: latex
%%% TeX-master: "main"
%%% End:

\begin{figure}

  \centering

  \begin{tikzpicture}
    \node[module]                        (I0) {\tiny Input};
    \node[module, right=3mm of I0]       (I1) {\tiny Parser (Z3)};
    \node[module, right=3mm of I1]       (I9) {\tiny Preprocessor};
    \node[module, right=3mm of I9]       (I8) {\tiny Congruence closure};
    \node[module, right=3mm of I8]       (I4) {\tiny Horn clause generator};
    \node[module, above=2mm of I4]       (I3) {\tiny EUF Interpolator};
    \node[module, below=5mm of I4, text width=20mm]  (I5) {\tiny Congruence closure with explanations};
    \node[module, below right=3mm of I4] (I6) {\tiny Simplifier (Z3)};
    \node[module, right=24mm of I4] (I7) {\tiny Output};
    \draw[->, dashed] (I4)--(I5);
    \draw[->, dashed] (I4)--(I6);
    \node[fit=(I8) (I4) (I5) (I6), draw, inner sep=0.5mm]           (fitThCombInternal) {};
    \node[fit=(I3) (fitThCombInternal), draw, inner sep=0.5mm] (fitThComb)         {};
    \draw[->] (I0)--(I1);
    \draw[->] (I1)--(I9);
    \draw[->] (I9)--(I8);
    \draw[->] (I8)--(I4);
    \draw[->] (I5)--(I4);
    \draw[->] (I6)--(I7);
  \end{tikzpicture}

  \caption{EUF Interpolator Diagram} \label{euf_architecture}

\end{figure}

%%% Local Variables:
%%% mode: latex
%%% TeX-master: "main"
%%% End:

\begin{figure}

  \centering

  \begin{tikzpicture}
    \node[module]                        (I0) {\tiny Input};
    \node[module, right=3mm of I0]       (I1) {\tiny Parser (Z3)};
    \node[module, right=3mm of I1]       (I9) {\tiny Preprocessor};
    \node[module, right=3mm of I9]       (I8) {\tiny Indexing table};
    \node[module, right=3mm of I8]       (I4) {\tiny Matching and Variable elimination};
    \node[module, above=2mm of I4]       (I3) {\tiny UTVPI Interpolator};
    \node[module, below=3mm of I4] (I6) {\tiny Simplifier};
    \node[module, right=3mm of I4] (I7) {\tiny Output};
    \draw[->, dashed] (I4)--(I6);
    \node[fit=(I8) (I4) (I6), draw, inner sep=0.5mm]           (fitThCombInternal) {};
    \node[fit=(I3) (fitThCombInternal), draw, inner sep=0.5mm] (fitThComb)         {};
    \draw[->] (I0)--(I1);
    \draw[->] (I1)--(I9);
    \draw[->] (I9)--(I8);
    \draw[->] (I8)--(I4);
    \draw[->] (I6)--(I7);
  \end{tikzpicture}

  \caption{UTVPI Interpolator Diagram} \label{utvpi_architecture}

\end{figure}

%%% Local Variables:
%%% mode: latex
%%% TeX-master: "main"
%%% End:


All the decision procedures mentioned in this chapter
were implemented with the exception of the SAT/SMT algorithms.
For the latter, PicoSAT/TraceCheck\cite{picosat08} and Z3 
\cite{10.1007/978-3-540-78800-3_24} were chosen as the 
libraries to work with these algorithms. 
The rest of this section discusses some minor modifications 
implemented in the above mentioned Z3 and the proof format 
output by PicoSAT/TraceCheck.

\subsection{Minor modifications to Z3}

Z3 standard input is SMTLib2 \cite{BarFT-SMTLIB}. This grammar
does not provide a standard specification regarding a suitable
format to work with interpolants. Interpolation software
read interpolant formulas based on the order of appereance
in a conjunction \cite{mcmillan2011interpolants}. In our case
we require two conjuncts of conjunctions of literals in the
EUF theory, UTVPI theory or combined theory. 

As we can notice in figures \ref{general_architecture}, \ref{euf_architecture},
and \ref{utvpi_architecture}, there is preprocessor component which prefixes
the names of uninterpreted symbols with the strings a\_, b\_, c\_
to indicate that the symbol name is either an A-local, B-local, or
common symbol respectively. We extended Z3's API with functions
that test if a formula is A-local, B-local, AB-pure, AB-common
based of the definitions in \cite{10.1007/11532231_26} because 
it is necessary to constantly check this conditions for splitting
purposes. Another reason for the latter is justified because the 
implemented congruence closure algorithm takes as an additional
criteria to maintain as representative term an AB-common term.
A similar change was implemented in the congruence closure implementation
of Z3. Nonetheless, it was irrelevant since Z3's internal structure
separates the abstract syntax tree, which is part of its API
with the enode data structures, which does not allow the super
to modify it. This is the reason why it was not
possible to work directly with Z3 congruence closure implementation
and a separate implementation was necessary.

\subsection{PicoSAT/TraceCheck Proof Format}

We used PicoSAT/TraceCheck to reconstruct a resolution-based proof 
necessary for Pudlak's algorithm in the interpolation combination 
component. Given that Z3 provides a user friendly proof-producing 
API \cite{Moura_proofsand}, why did the implementation
work require another SAT solver to obtain the resolution-proof?

There are several reasons for the latter. First, the author of the
thesis was not able to find an appropriate configuration of parameters
for the SMT solver to provide such proofs. In order to grasp an idea
of the latter, it was implemented a Z3 proof parser that render
a pdf rendered by \LaTeX. Many examples indicated that
Z3 selects more convenient theories to work with some problems. For
instance, the formula in pure propositional logic shown in figure 
\ref{query_proof_problem} used proof rules from EUF \footnote{Z3 uses
the term monotonicity instead of congruence}.

\begin{figure}
  \centering
  \includegraphics[scale=0.2]{query_screenshot}
  \caption{Problematic SMT query for resolution proofs} \label{query_proof_problem}
\end{figure}


%\includepdf[scale=0.2,pages=-,pagecommand={},width=\textwidth, angle=90]{python_proof.pdf}
\begin{figure}
  %\includepdf[scale=0.8,pages=-,angle=0]{python_proof.pdf}
  \begin{turn}{0}
\resizebox{420pt}{40pt}
{
  \centering
  \begin{prooftree}[separation=-0.5em]\infer0[asserted]{Or(Not(x1), x3)}\infer0[rewrite]{Or(Not(x1), x3)==Or(Not(x1), x3)}\infer2[mp]{Or(Not(x1), x3)}\infer0[rewrite]{Or(Not(x1), x3)==Or(Not(x1), x3)}\infer2[mp]{Or(Not(x1), x3)}\infer0[asserted]{x1}\infer1[iff-true]{x1==True}\infer1[monotonicity]{Not(x1)==Not(True)}\infer0[rewrite]{Not(True)==False}\infer2[trans]{Not(x1)==False}\infer0[asserted]{Or(Not(x1), Not(x3))}\infer0[asserted]{x1}\infer1[iff-true]{x1==True}\infer1[monotonicity]{Not(x1)==Not(True)}\infer0[rewrite]{Not(True)==False}\infer2[trans]{Not(x1)==False}\infer1[monotonicity]{Or(Not(x1), Not(x3))==Or(False, Not(x3))}\infer0[rewrite]{Or(False, Not(x3))==Not(x3)}\infer2[trans]{Or(Not(x1), Not(x3))==Not(x3)}\infer2[mp]{Not(x3)}\infer1[iff-false]{x3==False}\infer2[monotonicity]{Or(Not(x1), x3)==Or(False, False)}\infer0[rewrite]{Or(False, False)==False}\infer2[trans]{Or(Not(x1), x3)==False}\infer2[mp]{False}\end{prooftree}
}
\end{turn}

%%% Local Variables:
%%% mode: latex
%%% TeX-master: "main"
%%% End:

  \caption{Z3 proof of figure \ref{query_proof_problem}} \label{smt_proof_problem}
\end{figure}


Thus, we opted to use the PicoSAT SAT solver 
which implements the DPLL algorithm. 
PicoSAT, as many SAT solvers, take as input a 
DIMACS cnf file which consists of a
straightforward language to denote clauses. 
(Optional) Lines starting with the character `c'
denote comments, and a single line starting 
the string `p cnf' followed by
two positive integers denote the number of 
variables and number of clauses respectively.
The next lines encode clauses. Each line is a 
list of integers numbers terminating with
0. Positive number encode non-negated literals 
whose identifier is the number, and negative
numbers encode the negated literal whose 
identifier is the absolute value of the number.

The proof format produced by PicoSAT consists
of an ordered list of 
clauses which are either from the input problem 
(facts) or learned during the search. 
Each line is encoded following the DIMACS format,
but extended to include an identifying number at the
beginning of each line (which identifies the clause), and a 
second list of integers after the terminating zero of the clause 
which denotes the non-ordered list of clause identifiers 
involved in a hyper resolution step; the resulting clause 
obtained from this step is precisely the clauses encoded by
the line. An additional zero is used to terminate the second
list of integers.

\begin{example}
  The following extended proof trace is obtained after running
  the PicoSAT solver on the cnf file encoding the formula in example \ref{dpll_figure}:

  \begin{figure}[ht]
  \centering
  $
  \begin{matrix}
    1 & -1 &  0 &  0 &    & \\
    2 & -3 & 2 & 1 & 0 & 0  \\
    3 & 3 & 0 & 0 &  &      \\
    4 & 1 & -2 & 0 & 0 &    \\
    6 & -2 & 0 & 1 & 4 & 0  \\
    7 & 0 & 1 & 6 & 2 & 3   \\
  \end{matrix}
  $
  \caption{Example of extended proof trace by PicoSAT for
  problem \ref{dpll_figure}}
\end{figure}
As we can notice, the first four lines encode the original
input clauses; 1 denotes variable P, 2 denotes variable Q, and 3 denotes variable R. Also, these lines do not include
any clause identifier in the second list of number since
they are facts. The last two lines denote clauses with
identifier 6 and 7. Clause 6 is $\neg Q$, which is obtained
after the hyper resolution step using clauses 1 and 4. The
last line does not contain elements in the first list since
this encodes the contradiction clause.
\end{example}

From the previous example we can observe the list 
of clause identifiers for the hyper resolution step
are not ordered in the sense that we cannot produce
multiple resolution steps from the representation.
In order to obtain the latter one can implement a unit
propagation algorithm and discover such ordering.

TraceCheck is a tool incorporated in the previous version
of PicoSAT \cite{NiemetzPreinerBiere-JSAT15} to verify
proof traces. It is also able to obtain a pure 
resolution proof.
The presentation of the proof trace is restricted to 
contain only two clause identifiers in the second list
for each clause.

\begin{example}
  The following extended proof trace is obtained after running
  the TraceCheck on the previous proof trace:

  \begin{figure}[ht]
    \centering
    $
    \begin{matrix}
      3 &3& 0& 0& \\
      2 &1& 2& -3& 0& 0&\\
      4& 1& -2& 0& 0&\\
      1& -1& 0& 0&\\
      6& -2& 0& 1& 4& 0&\\
      7& 2& -3& 0& 1& 2& 0&\\
      8& -3& 0& 7& 6& 0&\\
      9& 0& 8& 3& 0&\\
    \end{matrix}
    $
    \caption{Example of extended proof trace by PicoSAT for
    problem \ref{dpll_figure}}
  \end{figure}
\end{example}


%%% Local Variables:
%%% mode: latex
%%% TeX-master: "main"
%%% End:
