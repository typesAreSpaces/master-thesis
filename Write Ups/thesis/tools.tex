\section{Algorithms}

TODO.

\subsection{DPLL, DPLL(T) and extensions}


TODO.

\subsection{Congruence Closure algorithms (DST, NO, with explanations)}

In \cite{10.1007/978-3-540-32033-3_33}, the authors introduced a Union-Find
data structure that supports the Explanation operation. This operation receives
as input an equation between constants. If the input equation is a consequence
of the current equivalence relation defined in the Union Find data structure, the Explanation
operation returns the minimal sequence of equations used to build such equivalence relation,
otherwise it returns `Not provable`. A proper implementation of this algorithm extends the traditional
Union-Find data structure with a \emph{proof-forest}, which consists of an additional
representation of the underlying equivalence relation that does not compress
paths whenever a call to the Find operation is made. For efficient reasons, the Find operation
uses the path compression and weighted union.

The main observation in \cite{10.1007/978-3-540-32033-3_33} is that, in order to
recover an explanation between two terms, by traversing the path between the two nodes
in the proof tree, the last edge in the path guarantees to be part of the explanation.
Intuitively, this follows because only the last Union operation was responsible of merging
the two classes into one. Hence, we can recursively recover the rest of the
explanation by recursively traversing the subpaths found.

Additionally, the authors in \cite{10.1007/978-3-540-32033-3_33} extended the
Congruence Closure algorithm \cite{10.1007/978-3-540-39813-4_5} using the above
data structure to provide Explanations for EUF theory. The congruence closure algorithm
is a simplification of the congruence closure algorithm in \cite{10.1145/322217.322228}.
The latter combines the traditional \emph{pending} and \emph{combine} list into one
single list, hence removing the initial \emph{combination} loop in the algorithm in
\cite{10.1145/322217.322228}.



TODO.

\subsection{Gallier}

In \cite{GALLIER1987233} it was proposed an algorithm for testing the unsatisfiability
of ground Horn clauses with equality. The main idea was to interleave two algorithms: \emph{implicational propagation}
(propositional satisfiability of Horn clauses) that updates the truth value of equations
in the antecedent of the input Horn clauses \cite{DOWLING1984267}; and \emph{equational propagation} (congruence closure
for grounded equations) to update the state of a Union-Find data structure \cite{10.1145/364099.364331}
that keeps the minimal equivalence relation defined by grounded equations in the input Horn clauses.

The author in \cite{GALLIER1987233} defined two variations of his algorithms by adapting
the Congruence Closure algorithms in \cite{10.1145/322217.322228, 10.1145/322186.322198}.
Additionally, modifications in the data structures used by the original algorithms were needed
to make the interleaving mechanism more efficient.

TODO.

\section{Available tools}


TODO.

\subsection{Z3 and my minor modifications}

TODO.

\subsection{zChaff and my minor modifications}

TODO.

\subsection{DIMACS}

TODO.

%%% Local Variables:
%%% mode: latex
%%% TeX-master: "main"
%%% End:
