\section{Conclusions}

This chapter presented the approach
by Prof. Kapur for computing
the uniform interpolant of a satisfiable
formula in the UTVPI theory.
Several examples and testing of the implementation
indicated that the output produced by Kapur's
algorithm produces a stronger interpolant 
than the other ones produced by iZ3 and Mathsat.
The performance comparison showed a closer
difference in times compared to the performance 
comparison obtained for the EUF theory. 

In order to improve the current implementation it
will be interesting to explore the use of heuristics to 
determine the order in which the uncommon variable 
should be eliminated. The current implementation uses
a lexicographic approach based on the occurrence of the
symbols in the input formula. Different heuristic
can consider the number of occurrences of the uncommon
symbols in different inequalities, a topological order
based on the dependency of the linear combinations of
the inequalities, etc. 


%%% Local Variables:
%%% mode: latex
%%% TeX-master: "main"
%%% End:
