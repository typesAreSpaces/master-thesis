\section{Conclusions}

This chapter presented the approach
by Prof. Kapur for computing
the uniform interpolant of an inconsistent
pair of formulas in the UTVPI theory.
Several examples and testing of the implementation
indicated that the output produced by Kapur's
algorithm produces a stronger interpolant 
than the othe ones produced by iZ3 and Mathsat.

The performance comparison showed a closer
difference in times compared to the performance 
comparison obtained for the EUF theory. 
It remains unclear why the performance comparison
data have a better quadratic fit when a linear
proof of the instance problem exists for 
all the algorithms. It will be interesting
to explore the reasons of the latter in the current
implementation for further improvements.

In order to improve the current implementation it
will be interesting to explore the use of heuristics to 
determine the order in which the uncommon variable 
should be eliminated. The current implementation uses
a lexicographic approach based on the occurance of the
symbols in the input formula. Different heuristic
can consider the number of ocurrences of the uncommon
symbols in different inequalities, a topological order
based on the dependecy of the linear combinations of
the inequalities, etc. 


%%% Local Variables:
%%% mode: latex
%%% TeX-master: "main"
%%% End:
