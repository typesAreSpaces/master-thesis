\chapter{Future Work}

Regarding implemenation work, the are several improvements that were not explored 
in the thesis work since many of them do not change the overall time complexity 
of the implementation, but these consume additional memory resources. 
These improvements are the following:

Major improvements:

\begin{itemize}
  \item Improve hash function for terms: during testing the congruence closure
    algorithm, it was evident that dealing with a large number of terms, collisions
    happened  to the point that some terms were merged since the signature
    function indicated to do so, but this merges should not happen. 
    Intrinsically, the current implementation relies on the hash function 
    provided by Z3, which might not be optimized
    for a large number of terms or it might
    be that their decision procedures
    encode internal information differently.
    Nonetheless, for
    a typical verification problem the implementation will not 
    find any problems since these
    issues were noticed while dealing with
    instance problems involving graph terms
    of more than 1000000 nodes.
  \item Regarding the theory combination 
    of EUF and UTPVI, it will interesting 
    to explore a
    more specific interpolation 
    combination approach. The thesis 
    culminated with a 
    uniform interpolant algorithm for the
    combine theory, but the author agrees 
    a better job can be done to avoid 
    propagating disjunctions of the equalities
    by extending the signature, 
    consequently the the interpolation 
    procedures, for both of the theories 
    involved or propagate these disjunctions as
    using a compact representation.
  \item Currently the implementation for combined EUF and UTVPI 
    theory does not incrementally check the 
    consistency of the current formula in the main loop of 
    the implementation \ref{thcomb_propagation_loop}. 
    Changing the latter to an incrementally approach will
    maintain lemmas so the computation of intermediate results
    will not be performed repetitively. 
\end{itemize}

Minor improvements:

\begin{itemize}
  \item The space needed to encode curry nodes \emph{can be significally 
    reduced} up to an order of \bigO{n}. The reason for 
    this current allocation schema is that it exists a 
    double bonding effect while performing curryification 
    of all the terms in the arguments of function applications.
\end{itemize}


%%% Local Variables:
%%% mode: latex
%%% TeX-master: "main"
%%% End:
