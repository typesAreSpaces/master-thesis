\chapter{Future Work}

Regarding implemenation work, the are several improvements that were not explored 
in the thesis work since many of them do not change the overal complexity 
of the implementation, but definitely these consume more resources.
These particular improvements are the following:

\begin{itemize}
  \item Improve hash function for terms: during testing the congruence closure
    algorithm, it was evident that dealing with a large of terms, collisions
    happened  to the point that some terms were merged since the signature
    function indicated to do so, but this should not happen. Nonetheless, for
    a typical verification problem the implementation will not find any problems.
  \item The space needed to encode curry nodes \emph{can be significally reduced
    } up to an order of \bigO{n}. The reason for this current allocation schema
    is that it exists a double bonding effect while performing curryification 
    of all the terms in the arguments of function applications.
\end{itemize}

Regarding potential extension of the systems, it will interesting to explore a
more specific interpolation combination approach as in \cite{10.1007/978-3-540-69738-1_25}.
The Nelson-Oppen framework seems to be adequate if the goal is to combine several theories
or be as general as possible. The reason why the hierarchical reasoning approach
was not implemented was because the authors did not make any claim about the possibility
to use non-convex theories. Perhaps, this is not a real limitation of the approach.

%%% Local Variables:
%%% mode: latex
%%% TeX-master: "main"
%%% End:
