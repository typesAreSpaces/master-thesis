\section{Interpolants}

TODO.

\subsection{Craig interpolation theorem}

Let $\alpha, \beta, \gamma$ be logical formulas in a given theory. If
$\models_{\mathcal{T}} \alpha \rightarrow \beta$, we say that $\gamma$ is an
interpolant for the pair $(\alpha, \beta)$ if the following conditions
are met:

\begin{itemize}
\item $\models_{\mathcal{T}} \alpha \rightarrow \gamma$
\item $\models_{\mathcal{T}} \gamma \rightarrow \beta$
\item Every non-logical symbol in $\gamma$ occurs both in $\alpha$ and
  $\beta$.
\end{itemize}

The \emph{interpolantion problem} can be stated naturally as 
follows: given two logical formulas $\alpha, \beta$ such that 
$\models_{\mathcal{T}} \alpha \rightarrow \beta$, find
the interpolant for the pair $(\alpha, \beta)$.

In his celebrated result \cite{10.2307/2963594}, Craig proved that for every pair
$(\alpha, \beta)$ of formulas in first-order logic such that
$\models \alpha \rightarrow \beta$, an interpolation formula exists. Nonetheless,
there are many logics and theories that this result does not hold \cite{komori1978}.

Usually, we see the interpolation problem defined differently in the literature, 
where it is considered $\beta^{'}$ to be $\neg \beta$ and 
the problem requires that the pair $(\alpha, \beta^{'})$
is mutually contradictory (unsatisfiable). This definition was popularized by 
McMillan \cite{10.1007/978-3-540-24730-2_2}. This shift of attention explains 
partially the further development in interpolation generation algorithms 
since many of these relied on SMT solvers that provided refutation proofs 
in order to (re)construct interpolants for different theories (and their 
combination) \cite{10.1007/978-3-642-02959-2_17, 
10.1007/978-3-642-36742-7_9, mcmillan2011interpolants}.

Extended definitions are given to the interpolation problem when dealing with specific
theories \cite{10.1007/11532231_26} in a way that interpreted function can be also part of
the interpolant. The latter is justify since otherwise, many interpolation formulas might
not exists in different theories (for example, lisp programs).

We also see different approaches to interpolation generation for particular
theories that exploits aspects of the underlying theory
\cite{10.1007/978-3-540-69738-1_25, 10.1007/11814771_21}.

TODO.

\subsection{Interpolants in verification (iZ3)}


TODO.

\section{Decision Procedures}

Given a theory $\mathcal{T}$ in a formula $\psi$ in 
the language of the theory, is it possible to know 
$\models_{\mathcal{T}} \psi$? The last question is 
known as the verification problem for the respective
theory $\mathcal{T}$. This question has been 
studied extensively for many theories of interest 
\cite{borger2001classical}. 

Regarding the decidability of the theories mentioned in Section \ref{math_theories}, we know that EUF is 
undecidable \cite{borger2001classical}, and the theory
of ordered commutative rings is undecidable when the
structure uses integers as the domain and the semantics
of the arithmetical operations \cite{DBLP:books/daglib/0076838} (interestingly this theory can be decidable if we
keep the same structure but use the reals as the domain
\cite{DBLP:books/daglib/0076838}).
Nonetheless, the quantifier-free fragment of EUF and 
the restriction imposed in the decision problem for 
the UTVPI theory allow efficient algorithms to decide 
validity and satisfiability in their respective theories 
\cite{10.1145/322186.322198, 10.1145/322217.322228, 10.1007/11559306_9}.

In the rest of this section we review some decision procedures used in
the implementation work of the thesis.

\subsection{Satisfiability and Satisfiability Modulo Theories}
 
The naive solution to satisfiability checking of a formula
consistent of enumerating all possible interpretations 
and assignments in a theory and test whether the last two are 
models for the formula. For the case of propositional logic 
we can see that such procedure with be \bigO{2^n}.

\subsection{Congruence Closure}

TODO.

\subsection{Satisfiability of Horn clauses of propositions and grounded equations}

TODO.

\subsection{Nelson-Oppen framework for theory combination}


TODO.

%%% Local Variables:
%%% mode: latex
%%% TeX-master: "main"
%%% End:
