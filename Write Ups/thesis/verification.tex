\section{Interpolants}

TODO.

\subsection{Craig interpolation theorem}

Let $\alpha, \beta, \gamma$ be logical formulas in a given theory. If
$\vdash \alpha \rightarrow \beta$, we say that $\gamma$ is an
interpolant for the pair $(\alpha, \beta)$ if the following conditions
are met:

\begin{itemize}
\item $\vdash \alpha \rightarrow \gamma$
\item $\vdash \gamma \rightarrow \beta$
\item Every non-logical symbol in $\gamma$ occurs both in $\alpha$ and
  $\beta$.
\end{itemize}

The \emph{interpolantion problem} can be stated naturally as follows: given two
logical formulas $\alpha, \beta$ such that $\vdash \alpha \rightarrow \beta$, find
the interpolant for the pair $(\alpha, \beta)$.

In his celebrated work \cite{10.2307/2963594}, Craig proved that for every pair
$(\alpha, \beta)$ of first order formulas such that
$\vdash \alpha \rightarrow \beta$, an interpolation formula exists.

Usually, we see the interpolation problem defined differently in the literature, where
we consider $\beta^{'}$ to be $\neg \beta$ and the problem requires that the pair $(\alpha, \beta^{'})$
is mutually contradictory (unsatisfiable). This definition was popularized by McMillan
\cite{10.1007/978-3-540-24730-2_2}. This shift of attention explains partially the further
development in interpolation generation algorithms since many of these relied on
SMT solvers that provided refutation proofs in order to (re)construct interpolants
for different theories (and their combination) \cite{10.1007/978-3-642-02959-2_17,
  10.1007/978-3-642-36742-7_9, mcmillan2011interpolants}.

Extended definitions are given to the interpolation problem when dealing with specific
theories \cite{10.1007/11532231_26} in a way that interpreted function can be also part of
the interpolant. The latter is justify since otherwise, many interpolation formulas might
not exists in different theories (for example, lisp programs).

We also see different approaches to interpolation generation for particular
theories that exploits aspects of the underlying theory
\cite{10.1007/978-3-540-69738-1_25, 10.1007/11814771_21}.

TODO.

\subsection{Interpolants in verification (iZ3)}


TODO.

\section{Decision Procedures}

Given a theory $\mathcal{T}$ in a formula $\psi$ in 
the language of the theory, is it possible to know $\models_{\mathcal{T}} \psi$? 
This general question is known as the verification problem for the respective
theory $\mathcal{T}$. This question has been studied extensively 
for many theories of interest \cite{börger2001classical}. 

The theories mentioned in the 

\subsection{Satisfiability and Satisfiability Modulo Theories}
 
The naive solution to satisfiability checking of a formula
consistent of enumerating all possible interpretations 
and assignments in a theory and test whether the last two are 
models for the formula. For the case of propositional logic 
we can see that such procedure with be \bigO{2^n}.

\subsection{Congruence Closure}

TODO.

\subsection{Satisfiability of Horn clauses of propositions and grounded equations}

TODO.

\subsection{Nelson-Oppen framework for theory combination}


TODO.

%%% Local Variables:
%%% mode: latex
%%% TeX-master: "main"
%%% End:
