\subsection{Illustrating 
Example}

Let us consider the following input formula in the
combined theory satisfying the weakening conditions in 
\ref{weakening_conditions}: $\{y - x \leq 0 , -y + x \leq 10
y + x \leq 20, -y - x \leq -10, -e + x \leq 0, e - y \leq 0, f(e) = x\}$ with symbols to eliminate to be $\{e\}$.

The normal form produced for the UTVPI component of the
proposed algorithm is the following
conjunction of inequalities:

$\{
  x \leq 15,
  -x \leq -5,
  y \leq 10,
  y + x \leq 20,
  y - x \leq 0,
  -y \leq 0,
  -y + x \leq 10,
  -y - x \leq -10,
  e \leq 10,
  e + x \leq 25,
  e - x \leq 5,
  e + y \leq 20,
  e - y \leq 10,
  -e \leq -5,
  -e + x \leq 10,
  -e - x\leq -10,
  -e + y \leq 5,
  -e - \leq -5
  \}$

Since the above contains $e \leq 10$ and $-e \leq -5$
then by rule 8. of the proposed algorithm we make 5 
branches propagating the equation $e = i$ where 
$i \in \{5, \dots, 10\}$.

The final output produced by the algorith is hence
$(f(5) = x \land \delta) \lor (f(6) = x \land \delta)
\lor (f(7) = x \land \delta)
\lor (f(8) = x \land \delta)
\lor (f(9) = x \land \delta)
\lor (f(10) = x \land \delta)
$ where $\delta$ is 
$x \leq 15 
\land -x \leq 5 
\land y \leq 10 
\land y + x \leq 20
\land y - x \leq 0
\land -y \leq 0
\land -y + x \leq 10
\land -y - x \leq -10$.

%%% Local Variables:
%%% mode: latex
%%% TeX-master: "main"
%%% End:
