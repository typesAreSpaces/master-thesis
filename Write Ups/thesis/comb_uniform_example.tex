\subsection{Illustrating 
Example}

Let us consider the following input formula in the
combined theory satisfying the weakening conditions in 
\ref{weakening_conditions}: $\{y - x \leq 0 , -y + x \leq 10,
y + x \leq 20, -y - x \leq -10, -e + x \leq 0, e - y \leq 0, f(e) = x\}$ with symbols to eliminate to be $\{e\}$.

The normal form produced for the UTVPI component of the
proposed algorithm is the following
conjunction of inequalities:

$\{
  x \leq 10,
  -x \leq -5,
  y \leq 10,
  y + x \leq 20,
  y - x \leq 0,
  -y \leq -5,
  -y + x \leq 0,
  -y - x \leq -10,
  e \leq 10,
  e + x \leq 20,
  e - x \leq 0,
  e + y \leq 20,
  e - y \leq 0,
  -e \leq -5,
  -e + x \leq 0,
  -e - x\leq -10,
  -e + y \leq 0,
  -e - y \leq -10
  \}$


Since the above contains $e - x \leq 0, e - y \leq 0, -e + x \leq 0, -e + y \leq 0$ 
then by rule 5. of the proposed algorithm we make 1
branch propagating the equation $e = x$.

The final output produced by the algorith is 
$f(x) = x \land x \leq 10
\land -x \leq -5 
\land y \leq 10 
\land y + x \leq 20
\land y - x \leq 0
\land -y \leq -5
\land -y + x \leq 0
\land -y - x \leq -10$

\subsection{Comparison of Craig Interpolants and Uniform Interpolants}

This section compares the Craig Interpolants 
obtained by Z3 and Mathsat for contradicting 
pairs of formulas and the uniform 
interpolant obtained by our implementation such that
the symbols to eliminate are the 
same for each input problem for our 
implementation.

\begin{table}[h]
  \centering
  \begin{tabular}{c|cccc}
    \toprule 

  \multicolumn{5}{c}{A-part: \makecell{$y - x \leq 0 \land -y + x \leq 10 \land
  y + x \leq 20 $ \\ $\land -y - x \leq -10 \land -e + x \leq 0\land e - y \leq 0 \land f(e) = x$}} \\

    \hline

    \multirow{6}{*}{SMT Solvers} & \multicolumn{4}{c}{B-parts} \\

    \cmidrule{2-5} \\

    {} & 
    \makecell{$x \geq 20$} & 
    \makecell{$x \neq y$} & 
    \makecell{$x \neq f(x)$} & 
    $x \neq f(y)$ \\

    \cmidrule{2-5} \\

    {} & \multicolumn{4}{c}{Interpolants} \\

    \hline

    Z3 & 
    \makecell{$x \leq 15$} & 
    \makecell{$
((0 \leq -x + y) \land $ \\ $((x = y) \lor (\neg (x - y \leq 0)))))
    $} & 
    \makecell{$f(x)=x$} & 
    \makecell{$
      (((\neg (y - x \leq 0)) $ \\ 
        $\lor (f(y) = x))$ \\ 
    $\land (x - y \geq 0))
    $} \\

    \hline

    Mathsat & 
    \makecell{$x \leq 15$} & 
    \makecell{$x \leq y \land y \leq x$} & 
    \makecell{$x = f(x)$} & 
    \makecell{$x = f(y)$}\\

    \hline

    Our implementation & \multicolumn{4}{c}{\makecell{$f(x) = x \land x \leq 10
\land -x \leq -5 
\land y \leq 10 
\land y + x \leq 20$ \\
$\land y - x \leq 0
\land -y \leq -5
\land -y + x \leq 0
\land -y - x \leq -10$}} \\

    \hline
  \end{tabular}
  \caption{Comparing Craig Interpolants and Uniform Interpolants for
  $\{y - x \leq 0 \land -y + x \leq 10 \land
  y + x \leq 20 \land -y - x \leq -10 \land -e + x \leq 0\land e - y \leq 0 \land f(e) = x, \{e\}\}$.}
\end{table}

Using Z3, we verified our implementation result implies the Craig interpolants obtained by Z3 and Mathsat.

%%% Local Variables:
%%% mode: latex
%%% TeX-master: "main"
%%% End:


%%% Local Variables:
%%% mode: latex
%%% TeX-master: "main"
%%% End:
