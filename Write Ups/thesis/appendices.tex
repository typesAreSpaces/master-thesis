\chapter*{Appendix}

\addcontentsline{toc}{chapter}{Appendices}

\section{Theorems about Interpolants}

\subsection{Interpolants are closed under conjunction and disjunction}

\begin{theorem}
  If $I_1, I_2$ are interpolants of $\langle A, B \rangle$, then 
  $I_1 \land I_2, I_1 \lor I_2$ are also interpolants of $\langle A, B \rangle$.
\end{theorem}

\begin{proof}
  We notice that $vars(I_1 \land I_2)
  \subseteq vars(A) \cap vars(B)$ and 
  $vars(I_1 \lor I_2) \subseteq vars(A) \cap vars(B)$,
  otherwise $I_1, I_2$ couldn't be interpolants.

  Since $I_1, I_2$ are interpolants, we have that 
  $A \vdash I_1$, $B \land I_1 \vdash \bot$ and 
  $A \vdash I_2$, $B \land I_2 \vdash \bot$

  Here are formal proofs for $A \vdash I_1 \land I_2$ and 
  $B \land I_1 \land I_2 \vdash \bot$:

  \begin{center}
    \begin{prooftree}
      \hypo{A}
      \ellipsis{}{I_1}
      \hypo{A}
      \ellipsis{}{I_2}
      \infer2{I_1 \land I_2}
    \end{prooftree}
    \qquad
    \begin{prooftree}
      \hypo{B \land I_1 \land I_2}
      \infer1{B \land I_1}
      \ellipsis{}{\bot}
    \end{prooftree}
  \end{center}

  Here are formal proofs for $A \vdash I_1 \lor I_2$ and 
  $B \land (I_1 \lor I_2) \vdash \bot$:

  \begin{center}
    \begin{prooftree}
      \hypo{A}
      \ellipsis{}{I_1}
      \infer1{I_1 \lor I_2}
    \end{prooftree}
    \qquad
    \begin{prooftree}
      \hypo{B \land (I_1 \lor I_2)}
      \ellipsis{*}{(B \land I_1) \lor (B \land I_2)}
      \infer0{B \land I_1}
      \ellipsis{}{\bot}
      \infer0{B \land I_2}
      \ellipsis{}{\bot}
      \infer3{\bot}
    \end{prooftree}
  \end{center}

  Where $*$ is any proof applying the distributivity property of the conjunction
  symbol over the disjunction symbol.

\end{proof}

\subsection{Interpolants distribute conjunctions over disjunctions in the A-part}

\begin{theorem} \label{interp_distribute}
  Let $\mathcal{F}_i$ be a set of formulas and $I_i$ an interpolant
  for each $\langle A \land \mathcal{F}_i, B \rangle$ respectively.
  Then $\bigvee_i I_i$ is an interpolant for 
  $\langle A \land (\bigvee_i \mathcal{F}_i), B \rangle$
\end{theorem}

\begin{proof}
  Let $A_i = A \land \mathcal{F}_i$ and $\hat{A} = A \land (\bigvee_i \mathcal{F}_i)$.

  We see that $vars(I_i) \subseteq vars(A_i) \subseteq vars(\hat{A})$, hence
  $vars(\bigvee_i I_i) = \bigcup_i vars(I_i) \subseteq \bigcup_i vars(A_i) 
  \subseteq vars(\hat{A})$. Similarly we can prove that $vars(\bigvee_i I_i) 
  \subseteq vars(B)$. Thus, $vars(\bigvee_i I_i) \subseteq 
  vars(A \land (\bigvee_i \mathcal{F}_i)) \cap vars(B)$.

  To prove: $\hat{A} \vdash \bigvee_i I_i$: We notice that $\hat{A} = \bigvee_i A_i$.
  From the latter and using the generalized version of the disjunction elimination 
  rule in logic, i.e.

  \begin{prooftree}
    \hypo{\alpha_1 \lor \dots \lor \alpha_n}

    \infer0{\alpha_1}
    \ellipsis{}{I_1}
    \infer1[]{I_1 \lor \dots \lor I_n}

    \hypo{\dots}

    \infer0{\alpha_n}
    \ellipsis{}{I_n}
    \infer1{I_1 \lor \dots \lor I_n}

    \infer4{I_1 \lor \dots \lor I_n}
  \end{prooftree}

  and distributing disjunctions over conjunctions in $\bigvee_i A_i$ the statement
  holds.

  To prove: $B \land \bigvee_i I_i \vdash \bot$: Since each $B \land I_i \vdash \bot$,
  by using the generalized version of the disjunction elimination rule from above, the
  result holds.

  Therefore, $\bigvee_i I_i$ is an interpolant for 
  $\langle A \land (\bigvee_i \mathcal{F}_i), B \rangle$.


\end{proof}

% Next lines duplicated from .toc file and used to create mini
% "Appendix Table of Contents," if desired:
% \contentsline {chapter}{\numberline {A}Proving $E=MC^2$}{4}
% \contentsline {chapter}{\numberline {B}Derivation of $A = \pi r^2$}{5}
% End mini table of contents

% \appendix
% \chapter{Proving $E=MC^2$}
%    I refer the reader to many of grandpa's famous books on this subject.
% \chapter{Derivation of $A = \pi r^2$}
%    A circle is really a square without corners.  QED.


%%% Local Variables:
%%% mode: latex
%%% TeX-master: "main"
%%% End:
