\section{Evaluation}

\subsection{Detailed evaluation of examples}

In this section we discuss in full detail the execution trace
of the implementation of some examples. 

\subsubsection{Symbol elimination example}

Let us consider the following example from \cite{KAPUR2017} 
$\alpha_1 = \{f(z_1, v) = s_1, f(z_2, v) = s_2, f(f(y_1, v), f(y_2, v)) = t\}$
with the set of symbols to eliminate $U_1 = \{u\}$. The implementation produces the following
trace (slightly modified for presentation purposes) in order to compute 
the interpolant for $\alpha_1; U_1$:

\verbatiminput{../../Software/Cpp/EUFInterpolant/tests/traces/current_progress_example1.txt}

The interface offered by SMT solvers with interpolation features usually provide the 
conventional A-part, B-part format. In order to compare the results with the implementation two 
instances were tested so we can obtain interpolants from both systems:

\begin{itemize}
\item Problem instance: A-part : $\{f(z_1, v) = s_1, f(z_2, v) = s_2, f(f(y_1, v), f(y_2, v)) = t\}$; B-part : $\{z_1 = z_2, s_1 \neq s_2 \}$
\begin{itemize}
\item Z3: (or (= s2 s1) (not (= z2 z1)))
\item Mathsat: (not (and (= z1 z2) (not (= s1 s2))))
\item Our implemenation: ($\rightarrow$ (= z1 z2) (= s1 s2))
\end{itemize}

\item Problem instance: A-part : $\{f(z_1, v) = s_1, f(z_2, v) = s_2, f(f(y_1, v), f(y_2, v)) = t\}$; B-part : $\{z_1 = y_1, z_1 = y_2, f(s_1, s_1) \neq t\}$
\begin{itemize}
\item Z3: (or (= (f s1 s1) t) (not (= z1 y1)) (not (= y2 y1)))
\item Mathsat: (not (and (not (= t (f s1 s1))) (and (= z1 y1) (= z1 y2))))
\item Our implemenation: ($\rightarrow$ (and (= y1 y2) (= z1 y2) (= z1 y2)) (= t (f s1 s1))))
\end{itemize}

\end{itemize}

Clearly, the interpolant obtained by just eliminating the symbol $v$ implies
the outputs produced by Z3 and Mathsat for the previous problem instances.

\subsubsection{Simple example with disequality}

Let us consider another example from \cite{KAPUR2017} 
$\alpha_2 = \{f(x_1) \neq f(x_2)\}$
with the set of symbols to eliminate $U_2 = \{f\}$. The implementation produces the following
trace for $\alpha_2; U_2$:

\verbatiminput{../../Software/Cpp/EUFInterpolant/tests/traces/current_progress_example2.txt}

To compare our result with Z3 and Mathsat we 
included the B-part formula to be $\{x_1 = x_2\}$.
The interpolants obtained by these systems were the 
same, which was (not (= x1 x2)).

%%% Local Variables:
%%% mode: latex
%%% TeX-master: "main"
%%% End:
