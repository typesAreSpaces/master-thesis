\section{Evaluation}

\subsection{Detailed evaluation of examples}

In this section we discuss in full detail the execution trace
of the implementation of some examples. 

\subsubsection{Symbol elimination example}

Let us consider the following example from \cite{KAPUR2017} 
$\alpha_1 = \{f(z_1, v) = s_1, f(z_2, v) = s_2, f(f(y_1, v), f(y_2, v)) = t\}$
with the set of symbols to eliminate $U_1 = \{u\}$. The implementation produces the following
trace (slightly modified for presentation purposes) in order to compute 
the interpolant of $\alpha_1; U_1$:

\verbatiminput{../../Software/Cpp/EUFInterpolant/tests/traces/current_progress_example1.txt}

The interface offered by SMT solvers with interpolation features usually provide the 
conventional A-part, B-part format. In order to compare the results with the implementation two 
instances were tested, so we can obtain interpolants from both systems:

\begin{itemize}
\item Problem instance: A-part : $\{f(z_1, v) = s_1, f(z_2, v) 
  = s_2, f(f(y_1, v), f(y_2, v)) = t\}$; B-part : 
  $\{z_1 = z_2, s_1 \neq s_2 \}$
\begin{itemize}
\item Z3: (or (= s2 s1) (not (= z2 z1)))
\item Mathsat: (not (and (= z1 z2) (not (= s1 s2))))
\item Our implementation: ($\rightarrow$ (= z1 z2) (= s1 s2))
\end{itemize}

\item Problem instance: A-part : $\{f(z_1, v) = s_1, f(z_2, v) 
  = s_2, f(f(y_1, v), f(y_2, v)) = t\}$; 
  B-part : $\{z_1 = y_1, z_1 = y_2, f(s_1, s_1) \neq t\}$
\begin{itemize}
\item Z3: (or (= (f s1 s1) t) (not (= z1 y1)) (not (= y2 y1)))
\item Mathsat: (not (and (not (= t (f s1 s1))) (and (= z1 y1) (= z1 y2))))
\item Our implementation: ($\rightarrow$ (and (= y1 y2) (= z1 y2) (= z1 y2)) (= t (f s1 s1))))
\end{itemize}

\end{itemize}

Clearly, the interpolant obtained by just eliminating the symbol $v$ implies
the outputs produced by Z3 and Mathsat for the previous problem instances.

\subsubsection{Simple example with dis-equality}

Let us consider another example from \cite{KAPUR2017} 
$\alpha_2 = \{f(x_1) \neq f(x_2)\}$
with the set of symbols to eliminate $U_2 = \{f\}$. The implementation produces the following
trace for $\alpha_2; U_2$:

\verbatiminput{../../Software/Cpp/EUFInterpolant/tests/traces/current_progress_example2.txt}

To compare our result with Z3 and Mathsat we 
included the B-part formula to be $\{x_1 = x_2\}$.
The interpolants obtained by these systems were the 
same, which was (not (= x1 x2)).

\subsection{Performance comparison with iZ3 and MathSat}\label{performance_euf}

This section discusses a benchmark for interpolant generation
for the EUF theory.
which it will allows us to test 
our implementation and constrast the execution with 
other interpolant generation
algorithms from Z3 and MathSat.

\subsubsection{Benchmark description}



KEEP:
we can also see that the formula $x = a \rightarrow \bot$ is an 
interpolating formula for all fixed $n$ in this parametrized problem.
We designed this problem because it is easy to verify the 
correctness of output of the algorithms and to measure 
the time used by the algorithms for large values of $n$. 
We executed instances of this problem for values of $n$
in the range $\{1, \dots, 10000\}$ using a computer desktop
equipped with an Intel i7-9700 @ 4.70 GHz and 16Gb of memory. 
The output produced by the interpolant generation algorithms
were the expected formula.
The following graph reports the time measured by the UNIX
utility $times$ of our implementation, iZ3, and the interpolation 
generation algorithm from Mathsat.

\begin{figure}
  \centering
  \includegraphics[scale=0.9]{figures/eufi_performance_graph}
  \caption{Performance comparison graph of EUF interpolant generation
  algorithms for paramatrized problem from section \ref{performance_euf}} 
  \label{performance_graph_euf}
\end{figure}

%%% Local Variables:
%%% mode: latex
%%% TeX-master: "main"
%%% End:


%%% Local Variables:
%%% mode: latex
%%% TeX-master: "main"
%%% End:
