\chapter{The Theory of UTVPI Formulas}

This theory appears heavily in formal methods dealing
with abstract domains introduced in \cite{journals/corr/abs-cs-0703084}.
The decision problem consists of checking the satisfiability
of a particular fragment of $LIA(\mathbb{Z})$. The fragment 
consists on conjunctions of inequalities with at most
two variables which integers coefficients are restricted
to $\{-1, 0, 1\}$. Efficient algorithms are found in the literature
for both the satisfiability problem \cite{10.1007/11559306_9} as well as 
for interpolation \cite{10.1007/978-3-642-02959-2_15} for this theory. Eventhough this fragment
looks severaly constraint, it has been used for a range
of applications where problems can be modelled using
this particular literals.

\section{Algorithm}

The algorithm proposed \cite{KAPUR2017} uses to inference rules
to close the relation and eliminate the uncommon variables from the
A-part of the input formula. The rules are the following:

\begin{center}
  \AxiomC{$s_1 x_1 + s_2 x_2 \leq c$}
  \RightLabel{Normalize}
  \UnaryInfC{$s_1 x_1 \leq \floor{\frac{c}{2}}$}
  \DisplayProof

  \AxiomC{$s_1 x_1 + s_2 x_2 \leq c_1$}
  \AxiomC{$- s_2 x_2 + s_3 x_3 \leq c_2$}
  \RightLabel{Elim}
  \BinaryInfC{$s_1 x_1 + s_3 x_3 \leq c_1 + c_2$}
  \DisplayProof
\end{center}

The algorithm normalizes the inequalities at the beginning as 
a preprocessing step and applies the Elim rule whenever it is
possible until no more uncommon variables remain in the 
input formula. Hence, having an efficient representation of 
the inequalities and detecting matches (like pivots for
resolution steps) is important for an efficient implementation.
To achieve this goal we implemented to an encoding of 
the inequalities using natural numbers, an array of 
numbers indexed by the numeral representation of the  inequalities
which keeps track of the minimum bound of the encoded inequality,
and a data structure to keep track of the signs of variables
in the inequalities for efficient matching.

%\begin{theorem}
  Given a mutually contradictory pair $(\alpha, \beta)$,
  where $\alpha, \beta$ are finite conjunctions of octagonal
  atoms, the above algorithm terminates with an interpolant
  $I_\alpha$ that is a finite conjunction of octagonal atoms
  and is equivalent to $\exists \vec{x}. \alpha$, where
  $\vec{x}$ is the symbols in $\alpha$ which are not in
  $\beta$. Further $I_\alpha$ is he strongest interpolant
  for $(\alpha, \beta)$.
\end{theorem}

\begin{proof}
  First we will prove that $\models \alpha \rightarrow I_\alpha$.
  The latter follows since the algorithm produces a
  conjunction of octagonal formulas using the Elim rule,
  which is a truth-preserving rule of inference, eliminating
  conjuncts with uncommon symbols.
  
  Now, we will prove that $\not\models I_\alpha \land \beta$. We
  will prove the latter by induction on the number of variables
  to eliminate ($k$).

  \begin{itemize}
  \item Base case: $k = 0$. Then the algorithm outputs $\alpha$.
    Then the statement holds since $(\alpha, \beta)$
    is unsatisfiable.
  \item Inductive case: $k = n + 1$. Since the set $S$ of variables to eliminate
    is non-empty, we just take any variable $x \in S$ and apply the
    above algorithm to eliminate such variable.
    Let $X$ be the set of octagonal inequalities of $(\alpha, \beta)$
    and $X^{'}$ be the set of octagonal inequalities $(\alpha^{'}, \beta)$
    where $\alpha^{'}$ is the conjunct obtained after removing the
    variable $x$. We know $X$ and $X^{'}$ are equisatisfiable using
    a similar argument as in the Fourier-Motzkin elimination
    method \cite{10.5555/17634}. Let us suppose
    $(\alpha^{'}, \beta)$ is satisfiable, hence $(\alpha, \beta)$
    is satisfiable as well. But the latter entails a contradiction
    since $(\alpha, \beta)$ is assumed to be unsatisfiable. Hence,
    $(\alpha^{'}, \beta)$ is unsatisfiable. Since $(\alpha^{'}, \beta)$
    is an unsatisfiable formula with $n$ variables to eliminate,
    using the Inductive Hypothesis we conclude that $(I_\alpha, \beta)$
    is unsatisfiable as well.
  \end{itemize}
  
\end{proof}

%%% Local Variables:
%%% mode: latex
%%% TeX-master: "main"
%%% End:


\section{Implementation}

In order to obtain a bijection between UTVPI inequalities and
natural numbers, first we define an ordering on the inequalities
and notice some invariants of the latter. 

We encode the term $\pm x_m \pm x_n$ using the point
$(\pm m, \pm n) \in \mathbb{Z}^2$. The variable $x_0$ is a
\emph{dummy variable} that acts as a place holder for 0. 
Additionally, we will restrict the terms/pairs such that the first
index variable is strictly greater \footnote{This condition also
  avoids the problem of keeping track of variable duplication in 
the inequality.} since addition is commutative.

We define the following ordering over the terms of the form
$\pm x_m \pm x_n$:

\begin{definition} Let $t_1 = s_1 x_{m_1} + s_2 x_{m_2}$ and
  $t_2 = s_1^{'} x_{n_1} + s_2^{'} x_{n_2}$. We say that 
  $t_1 > t_2$ if and only if or $s_1 = s_2 \land m_1 = m_2 and $
\end{definition}




The bijection between UTVPI inequalities and natural numbers
is given by the following procedure:

\section{Evaluation}
TODO.

%%% Local Variables:
%%% mode: latex
%%% TeX-master: "main"
%%% End:
