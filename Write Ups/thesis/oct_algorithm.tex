\section{Kapur's Uniform Interpolant Algorithm for
the UTVPI theory}

The algorithm proposed \cite{KAPUR2017} uses inference rules
to obtain the normal form of the conjunction of inequalities
of octagon formulas as well as 
eliminating the uncommon variables from the
A-part of the input formula. The rules are the following:

\begin{center}
  \begin{prooftree}
    \hypo[]{a x + a x \leq c}
    \hypo[]{a \in \{ -1, 0, 1 \}$ and $x \in Vars}
    \infer2[Normalize]{a x \leq \floor{\frac{c}{2}}}
  \end{prooftree}

  \begin{prooftree}
    \hypo[]{s_1 x_1 + s_2 x_2 \leq c_1}
    \hypo[]{- s_2 x_2 + s_3 x_3 \leq c_2}
    \infer2[Elim]{s_1 x_1 + s_3 x_3 \leq c_1 + c_2}
  \end{prooftree}
\end{center}

The algorithm normalizes the inequalities at the beginning as 
a preprocessing step and applies the Normalize and Elim rule 
whenever it is possible until no more uncommon variables 
remain in the input formula. Hence, having an efficient 
representation of 
the inequalities and match detection (similar to detecting pivots for
resolution in SAT) is important for an efficient implementation.

%\begin{theorem}
  Given a mutually contradictory pair $(\alpha, \beta)$,
  where $\alpha, \beta$ are finite conjunctions of octagonal
  atoms, the above algorithm terminates with an interpolant
  $I_\alpha$ that is a finite conjunction of octagonal atoms
  and is equivalent to $\exists \vec{x}. \alpha$, where
  $\vec{x}$ is the symbols in $\alpha$ which are not in
  $\beta$. Further $I_\alpha$ is he strongest interpolant
  for $(\alpha, \beta)$.
\end{theorem}

\begin{proof}
  First we will prove that $\models \alpha \rightarrow I_\alpha$.
  The latter follows since the algorithm produces a
  conjunction of octagonal formulas using the Elim rule,
  which is a truth-preserving rule of inference, eliminating
  conjuncts with uncommon symbols.
  
  Now, we will prove that $\not\models I_\alpha \land \beta$. We
  will prove the latter by induction on the number of variables
  to eliminate ($k$).

  \begin{itemize}
  \item Base case: $k = 0$. Then the algorithm outputs $\alpha$.
    Then the statement holds since $(\alpha, \beta)$
    is unsatisfiable.
  \item Inductive case: $k = n + 1$. Since the set $S$ of variables to eliminate
    is non-empty, we just take any variable $x \in S$ and apply the
    above algorithm to eliminate such variable.
    Let $X$ be the set of octagonal inequalities of $(\alpha, \beta)$
    and $X^{'}$ be the set of octagonal inequalities $(\alpha^{'}, \beta)$
    where $\alpha^{'}$ is the conjunct obtained after removing the
    variable $x$. We know $X$ and $X^{'}$ are equisatisfiable using
    a similar argument as in the Fourier-Motzkin elimination
    method \cite{Schrijver:1986:TLI:17634}. Let us suppose
    $(\alpha^{'}, \beta)$ is satisfiable, hence $(\alpha, \beta)$
    is unsatisfiable as well. But the latter entails a contradiction
    since $(\alpha, \beta)$ is assumed to be unsatisfiable. Hence,
    $(\alpha^{'}, \beta)$ is unsatisfiable. Since $(\alpha^{'}, \beta)$
    is an unsatisfiable formula with $n$ variables to eliminate,
    using the Inductive Hypothesis we conclude that $(I_\alpha, \beta)$
    is unsatisfiable as well.
  \end{itemize}
  
\end{proof}

%%% Local Variables:
%%% mode: latex
%%% TeX-master: "main"
%%% End:


%%% Local Variables:
%%% mode: latex
%%% TeX-master: "main"
%%% End:
