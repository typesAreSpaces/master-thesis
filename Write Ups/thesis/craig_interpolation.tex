\section{Interpolants}

Following the notation in \cite{10.1007/11532231_26}, we denote 
$\mathcal{V}(\psi)$ to be the set of non-logical symbols, variables
and constants of formula $\psi$. Given an instance for the interpolation
problem $(A, B)$ \footnote{For the rest of the thesis, we will denote the 
  first formula of an interpolation problem as the A-part 
and the second component as the B-part.}, 
we distinguish the following categories:

\begin{itemize}
  \item $\psi$ is \emph{A-local} if $\mathcal{V}(\psi) \subseteq 
    \mathcal{V}(A) \setminus (\mathcal{V}(A) \cap \mathcal{V}(B))$
  \item $\psi$ is \emph{B-local} if $\mathcal{V}(\psi) \subseteq
    \mathcal{V}(B) \setminus (\mathcal{V}(A) \cap \mathcal{V}(B))$
  \item $\psi$ is \emph{AB-common} if $\mathcal{V}(\psi) \subseteq
    \mathcal{V}(A) \cap \mathcal{V}(B)$
\item $\psi$ is \emph{A-pure} when either 
    $\mathcal{V}(\psi) \subseteq \mathcal{V}(A)$ 
\item $\psi$ is \emph{B-pure} when either 
    $\mathcal{V}(\psi) \subseteq \mathcal{V}(B)$ 
  \item $\psi$ is \emph{AB-pure} when either 
    $\psi$ is A-pure or $\psi$ is B-pure, otherwise
    $\psi$ is \emph{AB-mixed}
\end{itemize}

\begin{example} \label{first_example}

  Consider the following interpolation pair: $(f(a + 2) + 1 = c + 1
    \land f(a + 2) = 0
  , f(c) \leq b \land b < f(0))$. With respect to the previous 
  pair of formulas, we can tell that:
  \begin{itemize}
    \item The formula $f(a + 2) = c$ is 
      AB-pure but not A-local nor B-local nor AB-common
    \item The formula $\neg(a \leq f(f(b) + 1))$ is an AB-mixed
      literal
    \item The formula $a + 1 = 1$ is A-local.
    \item The formula $c = 0$ is AB-common.
    \item In general, $AB-common$ formulas are not $AB-pure$ formulas.
  \end{itemize}
\end{example}

\subsection{Craig interpolation theorem}

Let $\alpha, \beta, \gamma$ be logical formulas in a given theory. If
$\models_{\mathcal{T}} \alpha \rightarrow \beta$, we say that $\gamma$ is an
interpolant for the interpolation pair $(\alpha, \beta)$ if the following conditions
are met:

\begin{itemize}
  \item $\models_{\mathcal{T}} \alpha \rightarrow \gamma$
  \item $\models_{\mathcal{T}} \gamma \rightarrow \beta$
  \item Every non-logical symbol in $\gamma$ occurs both in $\alpha$ and
    $\beta$.
\end{itemize}

The \emph{interpolation problem} can be stated naturally as 
follows: given two logical formulas $\alpha, \beta$ such that 
$\models_{\mathcal{T}} \alpha \rightarrow \beta$, find
the interpolant for the pair $(\alpha, \beta)$.

In his celebrated result \cite{10.2307/2963594}, Craig proved that for every pair
$(\alpha, \beta)$ of formulas in first-order logic such that
$\models \alpha \rightarrow \beta$, an interpolation formula exists. Nonetheless,
there are many logics and theories for which this result does not hold \cite{komori1978}.

Usually, we see the interpolation problem defined 
differently in the literature, 
where it is considered $\beta^{'}$ to be $\neg \beta$ and 
the problem requires that the pair $(\alpha, \beta^{'})$
be mutually contradictory (unsatisfiable). 
This definition was popularized by 
McMillan \cite{10.1007/978-3-540-24730-2_2}. 
This shift of attention explains 
partially the further development in interpolation 
generation algorithms 
since many of these relied on SMT solvers that 
provided refutation proofs 
in order to (re)construct interpolants for different 
theories (and their 
combination) \cite{10.1007/978-3-642-02959-2_17, 
10.1007/978-3-642-36742-7_9, mcmillan2011interpolants}.

Relaxed definitions are considered for the interpolation 
problem when dealing with specific
theories \cite{10.1007/11532231_26}, permitting interpreted
functions to be included in the interpolant. The latter is 
justified since otherwise, many interpolation formulas might not exist
in different theories or the interpolants obtained might not 
be relevant (for example, lisp programs). This is formalized as follows:

\begin{definition} \cite{10.1007/11532231_26}
  Let $\mathcal{T}$ be a first-order theory of a signature $\Sigma$ and 
  let $\mathcal{L}$ be the class of quantifier-free $\Sigma$ formulas.
  Let $\Sigma_\mathcal{T} \subseteq \Sigma$ denote a designated set
  of interpreted symbols in $\mathcal{T}$. Let $A, B$ be formulas
  in $\mathcal{L}$ such that $A \land B \models_{\mathcal{T}} \bot$.
  A \emph{theory-specific} interpolant for $(A, B)$ in $\mathcal{T}$
  is a formula $I$ in $\mathcal{L}$ such that 
  $A \models_{\mathcal{T}} I$, $B \land I \models_{\mathcal{T}} \bot$,
  and $I$ refers only to AB-common symbols and symbols in 
  $\Sigma_{\mathcal{T}}$.
\end{definition}

\begin{example}
  In example \ref{first_example} we can tell $c + 1 = 1$ is not an
  interpolant simplify because the symbol $1$ only appears on the
  A-part. However, if $\Sigma_{\mathcal{LIA(\mathbb{Z})}}$ contains the
  interpreted symbols of $LIA(\mathbb{Z})$ (i.e. $+, *, 0, 1, 2, \dots$),
  then $c + 1 = 1$ becomes a \emph{theory-specific} interpolant. 

  Notice that $c = 0$ is an interpolant even if the set of interpreted symbols 
  used for interpolation is empty.
\end{example}

\subsection{Uniform Interpolant}

A uniform interpolant is a particular kind of interpolant for an inconsistent
pair of formulas. Introduced in \cite{pitts1992} as a 
construction to provide an interpretation
for second order intuitionistic propositional logic 
\emph{$IpC^2$}
\footnote{I.e. $IpC^2$ quantifies over 
propositional variables.}
using intuitionistic propositional logic \emph{IpC}.
Our notion of uniform interpolant is taken 
from \cite{ghilardi2020compactly}
where the authors provide the following definition:

\begin{definition}
  Given a theory $\mathcal{T}$ and an existential formula $\exists \cev{e} . \phi(\cev{e}, \cev{z})$, a quantifier-free formula 
  $\psi(\cev{y})$ is said to be a 
  $\mathcal{T}$\emph{-uniform interpolant} of 
  $\exists \cev{e} . \phi(\cev{e}, \cev{z})$ if

  \begin{itemize}
    \item $\mathcal{T} \models \exists 
      \cev{e} \phi(\cev{e}, \cev{y}) \rightarrow \psi(\cev{y})$
    \item $\psi(\cev{y})$ is the strongest interpolant
      (modulo $\mathcal{T}$), i.e. 
      if $\mathcal{T} \models \exists 
      \cev{e} \phi(\cev{e}, \cev{y}) \rightarrow \theta(\cev{z}, \cev{y})$, then $\mathcal{T} \models \psi(\cev{y}) \rightarrow \theta(\cev{z}, \cev{y})$.
  \end{itemize}

  A theory $T$ has the \emph{Uniform Interpolation Property} 
  if every existential formula
  $\exists \cev{e} . \phi(\cev{e},\cev{y})$ has a T-uniform interpolant. 

\end{definition}

\begin{example}
  Let us consider 
  $\alpha = \{f(x_1) \neq f(x_2)\}$
  with the set of symbols to eliminate $U = \{f\}$ in the
  EUF theory.

  We can prove that the EUF-uniform interpolant of
  $\alpha$ is $x_1 \neq x_2$. It is clear that 
  $f(x_1) \neq f(x_2) \models x_1 \neq x_2$. 
  Additionally, by the following lemma we show $x_1 \neq x_2$
  is the strongest interpolant:

  \begin{lemma}
    $x_1 \neq x_2$ implies any $\theta$ such that 
    $f(x_1) \neq f(x_2) \models \theta$
  \end{lemma}
  \begin{proof}
    By induction on the C.N.F. formula $\theta$
    \begin{itemize}
      \item Base case: $f(x_1) \neq f(x_2) \models x_1 \neq x_2 
        \land x_1 = x_1 \land x_2 = x_2$ and no other 
        literal involving $x_1, x_2$ in EUF are provable by
        $f(x_1) \neq f(x_2)$.
      \item Inductive step:
        \begin{itemize}
          \item Case $f(x_1) \neq f(x_2) \models \psi \land \phi$ :
            Thus, $f(x_1) \neq f(x_2) \models \psi$, $f(x_1) \neq f(x_2) \models \phi$.
            By IH. $x_1 \neq x_2 \rightarrow \psi$ and $x_1 \neq x_2 \rightarrow \phi$
            Therefore, $x_1 \neq x_2 \rightarrow \psi \land \phi$
          \item Case $f(x_1) \neq f(x_2) \models \psi \lor \phi$ : 
            We can use a congruence closure algorithm 
            to check if an equality belongs or not 
            in any EUF theory, thus we can check if 
            $f(x_1) \neq f(x_2) \models \psi$ or 
            $f(x_1) \neq f(x_2) \models \phi$. Let say W.L.O.G. 
            that $f(x_1) \neq f(x_2) \models \phi$. 
            By IH. $x_1 \neq x_2 \rightarrow \phi$.
            Thus, $x_1 \neq x_2 \rightarrow \psi \lor \phi$.
        \end{itemize}
    \end{itemize}
  \end{proof}

\end{example}

%%% Local Variables:
%%% mode: latex
%%% TeX-master: "main"
%%% End:
