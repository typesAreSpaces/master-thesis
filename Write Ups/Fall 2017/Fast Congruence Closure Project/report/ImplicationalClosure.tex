\section{Implicational Closure}

This section discusses the implicational closure algorithm in \cite{DOWLING1984267}.
The author of the previous paper refers to the problem of satisfiability of
propositional Horn Clauses. We use the term `implicational closure'
as the algorithm that solves the satisfiability problem of propositional Horn
Clauses. Later in the paper, we use implicational closure as part of other
algorithms.

It is well known from literature that the Boolean satisfiabity problem (SAT)
for propositional formulas is NP-complete \cite{Cormen:2009:IAT:1614191}.
Nonetheless, if we restrict the class of propositional formulas to Horn Clauses
then it is shown in \cite{DOWLING1984267} the existance of linear algorithm
capable to compute an interpretation for the propositional variables if the
formula is satisfiable, otherwise it halts returning `unsatisfiable formulas'
as a result. For a precise explanation of Horn Clauses, let us provide the
following definitions:

\definition{\cite{Leitsch:1997:RC:260906}} Let $\mathfrak{L}$ be a first-order
language. If $A$ is an \textit{atomic formula} then $A$ and $\neg A$ are
called \textit{literals}. An literal formula $A$ is called \textit{positive
literal} is $A$ is not of the form $\neg B$ for some atomic formula $B$.

\definition{\cite{Leitsch:1997:RC:260906}} A \textit{Horn clause} is a disjunction
of formulas containing a most one positive literal. A \textit{Horn formula} is a
conjunction of Horn clauses.

From the last definition, we can notice two trivial classes of Horn clauses:

\begin{itemize}
\item A does not contain a positive literal: Then $A$ is of the form $\neg P_1
  \lor \dots \lor \neg P_n$ with $n \geq 1$, where each $P_i$ is an atomic formula.
  Equivalently, we can express the above formula as $(P_1 \land \dots \land P_n)
  \rightarrow \bot$ where $bot$ is an unsatisfiable formula (bottom particle).
\item A contains one positive literal: Then $A$ is of the form $Q \lor \neg P_1
  \lor \dots \lor \neg P_n$ with $n \geq 0$, where each $P_i$ is an atomic
  formula. Similarly, we can express $A$ as $(P_1 \land \dots \land P_n)
  \rightarrow Q$.
\end{itemize}

We prefer the second `representation' of Horn clauses due to connections
with logic programming. For convenience, we take standard terminology from
logic programming including $body(H)$ to denote the set of formulae in the
precedent of a Horn clause $H$, and $head(H)$ as the formula in the consequent
of a Horn clause $H$. From this context, a Horn clause with exactly one
positive literal with $n = 0$ is called a `fact', and if $n > 0$ then the
Horn clause is denoted as a `rule'.

The key insight of Gallier's algorithm is precisely the observation mentioned
above about the relation between satisfiability of propositional Horn formulas
and implicational closure. The latter only implies the computation of the
following closure $\mathfrak{C}$:

\begin{itemize}
\item $\text{If } \vdash P \text{ then } P \in \mathfrak{C}$
  
\item $\text{If } P_1 \in \mathfrak{C}, P_2 \in \mathfrak{C}, \dots,
  P_n \in \mathfrak{C} \text{ and } (P_1 \land P_2 \land \dots
  \land P_n) \rightarrow Q \in \mathfrak{C} \text{ then } Q \in \mathfrak{C}$  
\end{itemize}

We notice that the latter closure $\mathfrak{C}$ can be understood as the set
that includes all `facts' of the Horn formula, and successively applies the
`\textit{Modus Ponens}' rules to include new facts from rules if possible.
Gallier's algorithm systematically does this process using a `queue' in order
to avoid unnecessary Modus Ponens calls. Indeed, his algorithm performs this
process partially. For each atomic formula, the author includes
atomic formulae extra variable to keep track of its current truth value.

\begin{itemize}
\item[Step 1.] Initially, all truth values for atomic formulae are set to \texttt{false} 
  and changed only if such atomic formulae are facts in the Horn formula.
\item[Step 2.] Any atomic formula that has changed its truth values from \texttt{false} to
\texttt{true} are pushed into a queue. Two additional data structures keep track
of which Horn clauses include a given atomic formula, and the number of unsatisfied
atomic formulae in the body for each Horn clause. Intuitively, if the number of
unsatisfied formulae in a Horn clause is zero means that the head of the Horn clause
should be included in the closure.
\item[Step 3.] If the queue is empty, then halt and indicate that the Horn formula is
  satisfiable; the truth assignment for each atomic formula is already stored in the
  extended representation mentioned above. Otherwise, pop the first element
  from the queue
  and repeat Step 2.
\end{itemize}

From the previous description of Gallier's algorithm we can see that the
time required is $\bigO{n}$ where $n$ is the number of atomic formulae in the
Horn formula since any atomic formula is never pushed more than once into the
queue.